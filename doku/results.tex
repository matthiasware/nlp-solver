\chapter{Measurements and Results}\label{ch:results}
\section{Measurements}
We tested our implementation on a subset of the CUTEst problem set. For this we selected only bounded and unbounded problems with variable-size $2 \leq n \leq 5000$. In total $309$ problems were used for testing. Throughout the test runs, the following solver configuration was employed
\begin{flalign*}
	f_{tol} = 1e^{-8} \; \text{(see~\eqref{eq:ftol}) }
	\qquad
	g_{tol} = 1e^{-8} \; \text{(see~\eqref{eq:gtol})}
\end{flalign*}
We set the limit for the number of iterations at
\begin{flalign*}
	k_{max} = 5000
\end{flalign*}
and confined the CPU runtime of the solving process to $360$ seconds.
The variables of interest were:
\begin{itemize}
	\item \textbf{Success}: Was the problem solved or unsolved.
	\item \textbf{Function value}: The objective function value of the last iterate, which is the minimum if the solver terminated successfully, or the last iterate when the solver was aborted due to the time cap, or the limit on the number of iterations.
	\item \textbf{Iterations}: Number of iterations performed until the solver terminated.
	\item \textbf{Message} The result message of the solver.
\end{itemize}
The summary of the NOONTIME test run is listed in Table~\ref{tab:stat:noontime}. \\

For comparison, we used the open source solver library IPopt (version 3.12.5)  ~\cite{webpage:ipopt}. IPopt as part of the COIN-OR initiative is written in C++ and primarily optimized for large-scale optimization problems.
It was initially released in 2005 and since then steadily improved. It is well recognized in both academics and industry ~\cite{webpage:ipopt}.
This and the fact that IPopt uses Newton's method, makes it a good competitor for NOONTIME.
We ran IPopt with its default configuration on the same problem set.
A summary of the Ipopt test run is listed in Table~\ref{tab:stat:ipopt}. \\

The results of all the test runs for both, NOONTIME and IPopt can be found in Table~\ref{tab:fullresults}. \\

For comparing the results from both of the solvers we use the \textit{relative solver error}. Here we introduce the \textit{relative solver error} on the values of the objective functions, but it is similarly defined and used for the number of iterations. Given a problem from our test set with results for the objective function value from NOONTIME, $f_{nt}$, and from IPopt, $f_{opt}$.
Since the absolute error of both results $|f_{nt} - f_{pt} |$ depends heavily on the problem and does not represent the goodness of the final results very well, we define the \textit{relative solver error} for $f_{nt}$ and $f_{opt}$ as:

\begin{flalign}
err_{rel}(i) = \frac{f_i - f_{min}}{|f_{min}| + 1} \qquad f_{min} = min(f_{opt}, f_{nt}) \quad i \in \{nt, opt\}
\end{flalign}
This allows us to classify the result. With $\epsilon$ being appropriately selected, we denote three classes as:
\begin{flalign}
&\!\begin{aligned}
f_{opt} \ll f_{nt} \quad \Leftrightarrow \quad
& f_{opt} < f_{nt} \text{ and } err_{rel}(nt) > \epsilon \\
f_{opt} \approx f_{nt} \quad \Leftrightarrow \quad
& f_{nt} \leq f_{opt} \text{ and }  err_{rel}(opt) < \epsilon \text{ or } \\
& f_{opt} \leq f_{nt} \text{ and } err_{rel}(nt) < \epsilon \\
f_{opt} \gg f_{nt} \quad \Leftrightarrow \quad 
& f_{opt} < f_{nt} \text{ and } err_{rel}(opt) > \epsilon
\end{aligned}
\label{eq:def:goodness}
\end{flalign}
Informally, these classes can be interpreted verbally as:
\begin{itemize}
	\item $f_{opt} \approx f_{nt}$: IPopt and NOONTIME did equally well.
	\item $f_{opt} \ll f_{nt}$: IPopt did better than NOONTIME.
	\item $f_{opt} \gg f_{nt}$: NOONTIME did better than IPopt.
\end{itemize}

The same holds for the number of iterations of IPopt, $Iter_{opt}$, and NOONTIME, $Iter_{nt}$. For the evaluation of our results, the relative solver error was computed with the following parameters:
\begin{itemize}
	\item Objective function value: $\epsilon = 1e^{-4}$
	\item Number of iterations: $\epsilon = 1$
\end{itemize}
In the full result Table~\ref{tab:fullresults}, we color-encoded these classes for both the number of iterations and the objective function values.
Let $f_{opt} \geq f_{nt}$. Then the cell for $f_{opt}$ is colored \textit{dark green}. If $f_{opt} = f_{nt}$, then also the cell for $f_{nt}$ is colored \textit{dark green}. Otherwise if $f_{nt} \approx f_{opt}$, then it is colored \textit{light green} and if $f_{opt} \ll f_{nt}$, it is colored \textit{orange}.
Informally, a cell is \textit{dark green} if it holds the best result for this specific problem, it is \textit{light green} if the result is similarly good as the best result and it is \textit{orange} if it is sufficiently worse than the best result.
The same rules apply to the columns of \textit{\#iter}.

\begin{table}[H]
	\begin{tabular}{cccl}
		\multicolumn{4}{c}{\textit{NOONTIME}} \\
		\toprule
		\toprule
		\textbf{status} & \textbf{code} & \textbf{count} & \textbf{reason} \\
		\midrule
		solved & 0 & 241 & Optimal Solution Found. \\
		\midrule
		\multirow{6}{*}{unsolved}
		& 1 & 16 & Maximum number of iterations exceeded. \\
		& 2 & 40 & Timeout after 360 seconds. \\
		& 5 & 9 & Invalid iterate encountered ($f^{(k+1)} > f^{(k)}$). \\
		& 6 & 1 & Overflow encountered in double scalars. \\
		& 7 & 1 & Eigenvalues did not converge. \\
		& 8 & 1 & Invalid search direction encountered ($g^T \Delta x > 0)$. \\
		\bottomrule
		& $\Sigma$ & $309$ &  \\
		\bottomrule
	\end{tabular}
	\caption{Result breakdown of running NOONTIME on the CUTEst test set.}
	\label{tab:stat:noontime}
\end{table}

\begin{table}[H]
\begin{tabular}{cccl}
	\multicolumn{4}{c}{\textit{IPopt}} \\
	\toprule
	\toprule
	\textbf{status} & \textbf{code} & \textbf{count} & \textbf{reason} \\
	\midrule
	solved & 0 & 290 & Optimal Solution Found. \\
	\midrule
%	\multirow{4}{*}{\rotatebox[origin=c]{90}{unsolved}}
	\multirow{4}{*}{unsolved}
	& 1 & 11 & Maximum Number of Iterations exceeded. \\
	& 2 & 5 & Timeout after 360 seconds. \\
	& 3 & 1 & Invalid number in NLP function or derivative detected. \\
	& 4 & 2 & Error in step computation. \\
	\bottomrule
	& $\Sigma$  &  $309$ & \\
	\bottomrule
\end{tabular}
\caption{Result breakdown of running IPopt on the CUTEst test set.}
\label{tab:stat:ipopt}
\end{table}

\begin{table}[H]
	\begin{tabular}{ccccc}
		\multicolumn{4}{c}{\textit{Problems solved by NOONTIME and IPopt}} \\
		\toprule
		\toprule
		& \multicolumn{3}{c}{\textbf{Function value}} \\
		\cmidrule{2-4}
		\textbf{Iterations} & $f_{opt} \ll f_{nt}$ & $f_{opt} \approx f_{nt}$  &  $f_{opt} \gg f_{nt}$ & $\Sigma$\\
		\midrule
		%	\multirow{4}{*}{\rotatebox[origin=c]{90}{Iterations}} 
		$Iter_{opt} \ll Iter_{nt}$  & 4 & 27 & 0 & 31 \\
		$Iter_{opt} \approx Iter_{nt}$ & 6 & 160 & 3 & 169 \\
		$Iter_{opt} \gg Iter_{nt}$ & 7 & 24 & 4 & 35 \\
		\midrule
		$\Sigma$ & 17 & 211 & 7 & 235  \\
		\bottomrule
	\end{tabular}
	\caption{Comparing the results from all problems, that both, IPopt and NOONTIME solved.} 
	\label{tab:ipopt_vs_noontime_solved}
\end{table}


\section{Evaluation}
On the given $309$ problems, NOONTIME solved $241$, which amounts a success rate of $77.99 \%$ and IPopt in comparison solved $290$ problems, equivalent to $93.94 \%$. The relationships of results regarding the success variable are listed in Table~\ref{tab:noontime_vs_ipopt_total}.

\begin{table}[H]
	%\noindent
	\begin{center}
		\renewcommand\arraystretch{1.5}
		\setlength\tabcolsep{0pt}
		\begin{tabular}{c >{\bfseries}r @{\hspace{0.7em}}c @{\hspace{0.4em}}c @{\hspace{0.7em}}l}
			\multirow{10}{*}{\parbox{1.1cm}{\bfseries\raggedleft IPopt}} & 
			& \multicolumn{2}{c}{\bfseries NOONTIME} & \\
			& & \bfseries solved & \bfseries unsolved & \bfseries $\Sigma$ \\
			& solved & \MyBoxx{235}{} & \MyBoxx{55}{} & 290 \\[2.4em]
			& unsolved & \MyBoxx{6}{} & \MyBoxx{13}{} & 19 \\
			& $\Sigma$ & 241 & 68 & 309
		\end{tabular}
	\end{center}
	\caption{Results of testing NOONTIME and IPopt on the problem set.}
	\label{tab:noontime_vs_ipopt_total}
\end{table}

\subsection{Unsolved problems}
For NOONTIME, the largest group of unsolved problems is the group with code number $2$ where the solver process was aborted due to the cut-off time. It counts $40$ problems in total. By comparison, for IPopt the same group contains only $5$ members. This gap in performance can be explained easily:
\footnote{We note here, that we did not optimize for speed in terms of CPU time, which is also the reason why we did not measure and compare the runtimes in the results.}
\begin{enumerate}
	\item \textbf{Sparsity}: In contrast to IPopt, NOONTIME does not exploit sparsity structures in the Hessian matrix. Especially for problems with many variables, this slows down the solving process with expensive operations like eigenvalue computation or the solving of linear systems.

	\item \textbf{Native Python}: The Cauchy point computation and the subspace minimization consist of many loops that are implemented in native Python which runs slower in comparison to a compiled version of the same.
\end{enumerate}

The second largest group (code number $1$) of unsolved problems with $16$ members are the problems where the maximum number of iterations was exceeded. Here we are not much worse off than IPopt with $11$ problems in this category. Exploratory test runs with modified solver configuration settings suggests, that the size of this group can be reduced by adjusting the solver configurations to the specific problems. \\

The three problems with exit code 6,7 and 8 in Table~\ref{tab:stat:noontime} could not be solved because of numerical issues in our implementation.
The problems with exit code 5 in Table~\ref{tab:stat:noontime} could not be solved due to numerical issues in the \textit{scipy} line search that we rely on. \\

\subsection{Solved problems}

As shown in Table~\ref{tab:noontime_vs_ipopt_total} there are $235$ problems that both IPopt and NOONTIME solved. We classified the results of these problems according to the criterion~\eqref{eq:def:goodness} which is summarized in Table~\ref{tab:ipopt_vs_noontime_solved}. \\
From the $235$ there are $191$ problems or $81,3\%$ where NOONTIME was at least as good as IPopt, both in terms of the goodness of the minimum and the number of iterations. In contrast, for IPopt there are $197$ problems or $83,9\%$, where it was at least as good as NOONTIME.

Now considering the $211$ problems where $f_{opt} \approx f_{nt}$, we were interested in how fast (in terms of iterations) the two implementations converged. By correlating the relative number of problems solved with the relative overhead in terms of iterations as shown in in Figure~\ref{fig:iterations}, we conclude, that both solvers behave similarly regarding convergence speed.

\begin{figure}[H]
	\includegraphics[width=\textwidth]{img/iterations.png}
	\caption{Convergence speed for the intersection of problems that were solved by IPopt and NOONTIME. The blue curve represents NOONTIME and the orange one IPopt.}
	\label{fig:iterations}
\end{figure}

\chapter{Appendix}
{\setlength\tabcolsep{3.0pt}\footnotesize  % default value: 6pt
	\begin{longtable}{ccccccccccc}
	\label{tab:fullresults}\\
	\toprule
%	\multicolumn{1}{p{3cm}}{\cellcolor{header} \centering Problem \\ name}&
	\thead{\textit{name}}&
	\thead{\textit{bounds}}&
	\thead{\textit{n}}&
	\thead{\textit{success} \\ {\tiny noontime}}&
	\thead{\textit{success} \\ {\tiny ipopt}}&
%	\multicolumn{1}{p{1.2cm}}{\cellcolor{header} \centering success \\ noontime} &
%	\cellcolor{header} success.i&
%	\multicolumn{1}{p{1.2cm}}{\cellcolor{header} \centering success \\ ipopt} &
	\thead{$f$ \\ {\tiny noontime}}& 
%	\multicolumn{1}{c}{\cellcolor{header} \centering fvalue \\ noontime} &
	\thead{$f$ \\ {\tiny ipopt}}&
	\thead{\textit{\#iter} \\ {\tiny noontime}}&
	\thead{\textit{\#iter} \\ {\tiny ipopt}}&
	\thead{\textit{code} \\ {\tiny noontime}}&
	\thead{\textit{code} \\ {\tiny ipopt}} \\
	\midrule
	\endhead
	\cellcolor{default1} 3PK& \cellcolor{default1} True& \cellcolor{default1} 30& \cellcolor{default1} True& \cellcolor{default1} True& \cellcolor{best} 1.720119E+00& \cellcolor{ok} 1.720119E+00& \cellcolor{best} 1& \cellcolor{poor} 11& \cellcolor{default1} 0& \cellcolor{default1} 0\\
	\cellcolor{default2} AIRCRFTB& \cellcolor{default2} True& \cellcolor{default2} 8& \cellcolor{default2} True& \cellcolor{default2} True& \cellcolor{ok} 1.890693E-08& \cellcolor{best} 4.788247E-25& \cellcolor{poor} 58& \cellcolor{best} 15& \cellcolor{default2} 0& \cellcolor{default2} 0\\
	\cellcolor{default1} AKIVA& \cellcolor{default1} False& \cellcolor{default1} 2& \cellcolor{default1} True& \cellcolor{default1} True& \cellcolor{ok} 6.166042E+00& \cellcolor{best} 6.166042E+00& \cellcolor{best} 6& \cellcolor{best} 6& \cellcolor{default1} 0& \cellcolor{default1} 0\\
	\cellcolor{default2} ALLINIT& \cellcolor{default2} True& \cellcolor{default2} 4& \cellcolor{default2} True& \cellcolor{default2} True& \cellcolor{best} 1.670597E+01& \cellcolor{ok} 1.670597E+01& \cellcolor{best} 7& \cellcolor{ok} 11& \cellcolor{default2} 0& \cellcolor{default2} 0\\
	\cellcolor{default1} ALLINITU& \cellcolor{default1} False& \cellcolor{default1} 4& \cellcolor{default1} True& \cellcolor{default1} True& \cellcolor{best} 5.744385E+00& \cellcolor{ok} 5.744385E+00& \cellcolor{best} 7& \cellcolor{ok} 14& \cellcolor{default1} 0& \cellcolor{default1} 0\\
	\cellcolor{default2} ARGLINA& \cellcolor{default2} False& \cellcolor{default2} 200& \cellcolor{default2} True& \cellcolor{default2} True& \cellcolor{best} 2.000000E+02& \cellcolor{best} 2.000000E+02& \cellcolor{best} 1& \cellcolor{best} 1& \cellcolor{default2} 0& \cellcolor{default2} 0\\
	\cellcolor{default1} ARGLINB& \cellcolor{default1} False& \cellcolor{default1} 200& \cellcolor{default1} True& \cellcolor{default1} True& \cellcolor{best} 9.962547E+01& \cellcolor{ok} 9.962547E+01& \cellcolor{best} 2& \cellcolor{best} 2& \cellcolor{default1} 0& \cellcolor{default1} 0\\
	\cellcolor{default2} ARGLINC& \cellcolor{default2} False& \cellcolor{default2} 200& \cellcolor{default2} True& \cellcolor{default2} True& \cellcolor{best} 1.011255E+02& \cellcolor{ok} 1.011255E+02& \cellcolor{best} 2& \cellcolor{best} 2& \cellcolor{default2} 0& \cellcolor{default2} 0\\
	\cellcolor{default1} ARGTRIGLS& \cellcolor{default1} False& \cellcolor{default1} 10& \cellcolor{default1} True& \cellcolor{default1} True& \cellcolor{ok} 3.762173E-19& \cellcolor{best} 7.054688E-25& \cellcolor{best} 5& \cellcolor{ok} 8& \cellcolor{default1} 0& \cellcolor{default1} 0\\
	\cellcolor{default2} ARWHEAD& \cellcolor{default2} False& \cellcolor{default2} 5000& \cellcolor{default2} True& \cellcolor{default2} True& \cellcolor{best} 0.000000E+00& \cellcolor{best} 0.000000E+00& \cellcolor{best} 6& \cellcolor{best} 6& \cellcolor{default2} 0& \cellcolor{default2} 0\\
	\cellcolor{default1} BA-L1LS& \cellcolor{default1} False& \cellcolor{default1} 57& \cellcolor{default1} True& \cellcolor{default1} True& \cellcolor{best} 1.256169E-24& \cellcolor{ok} 7.648105E-21& \cellcolor{poor} 138& \cellcolor{best} 10& \cellcolor{default1} 0& \cellcolor{default1} 0\\
	\cellcolor{default2} BA-L1SPLS& \cellcolor{default2} False& \cellcolor{default2} 57& \cellcolor{default2} True& \cellcolor{default2} True& \cellcolor{best} 2.105604E-23& \cellcolor{ok} 6.476196E-17& \cellcolor{poor} 23& \cellcolor{best} 9& \cellcolor{default2} 0& \cellcolor{default2} 0\\
	\cellcolor{default1} BARD& \cellcolor{default1} False& \cellcolor{default1} 3& \cellcolor{default1} True& \cellcolor{default1} True& \cellcolor{ok} 8.214877E-03& \cellcolor{best} 8.214877E-03& \cellcolor{best} 8& \cellcolor{best} 8& \cellcolor{default1} 0& \cellcolor{default1} 0\\
	\cellcolor{default2} BDEXP& \cellcolor{default2} True& \cellcolor{default2} 5000& \cellcolor{default2} False& \cellcolor{default2} True& \cellcolor{poor} 7.652549E-04& \cellcolor{best} 1.612155E-06& \cellcolor{poor} 39& \cellcolor{best} 18& \cellcolor{default2} 2& \cellcolor{default2} 0\\
	\cellcolor{default1} BDQRTIC& \cellcolor{default1} False& \cellcolor{default1} 5000& \cellcolor{default1} True& \cellcolor{default1} True& \cellcolor{best} 2.000626E+04& \cellcolor{ok} 2.000626E+04& \cellcolor{ok} 10& \cellcolor{best} 9& \cellcolor{default1} 0& \cellcolor{default1} 0\\
	\cellcolor{default2} BEALE& \cellcolor{default2} False& \cellcolor{default2} 2& \cellcolor{default2} True& \cellcolor{default2} True& \cellcolor{ok} 2.895403E-16& \cellcolor{best} 4.342571E-18& \cellcolor{best} 6& \cellcolor{ok} 8& \cellcolor{default2} 0& \cellcolor{default2} 0\\
	\cellcolor{default1} BENNETT5LS& \cellcolor{default1} False& \cellcolor{default1} 3& \cellcolor{default1} True& \cellcolor{default1} True& \cellcolor{best} 5.389879E-04& \cellcolor{ok} 5.563289E-04& \cellcolor{poor} 90& \cellcolor{best} 21& \cellcolor{default1} 0& \cellcolor{default1} 0\\
	\cellcolor{default2} BIGGS3& \cellcolor{default2} True& \cellcolor{default2} 6& \cellcolor{default2} True& \cellcolor{default2} True& \cellcolor{ok} 6.490822E-09& \cellcolor{best} 4.080557E-17& \cellcolor{poor} 118& \cellcolor{best} 9& \cellcolor{default2} 0& \cellcolor{default2} 0\\
	\cellcolor{default1} BIGGS5& \cellcolor{default1} True& \cellcolor{default1} 6& \cellcolor{default1} True& \cellcolor{default1} True& \cellcolor{ok} 1.724764E-07& \cellcolor{best} 4.016507E-17& \cellcolor{poor} 64& \cellcolor{best} 20& \cellcolor{default1} 0& \cellcolor{default1} 0\\
	\cellcolor{default2} BIGGS6& \cellcolor{default2} False& \cellcolor{default2} 6& \cellcolor{default2} True& \cellcolor{default2} True& \cellcolor{ok} 1.579136E-09& \cellcolor{best} 3.748460E-17& \cellcolor{poor} 320& \cellcolor{best} 79& \cellcolor{default2} 0& \cellcolor{default2} 0\\
	\cellcolor{default1} BIGGSB1& \cellcolor{default1} True& \cellcolor{default1} 5000& \cellcolor{default1} False& \cellcolor{default1} True& \cellcolor{poor} 1.558711E-01& \cellcolor{best} 1.500460E-02& \cellcolor{best} 16& \cellcolor{ok} 17& \cellcolor{default1} 2& \cellcolor{default1} 0\\
	\cellcolor{default2} BLEACHNG& \cellcolor{default2} True& \cellcolor{default2} 17& \cellcolor{default2} False& \cellcolor{default2} True& \cellcolor{best} 9.176758E+03& \cellcolor{ok} 9.176759E+03& \cellcolor{best} 4& \cellcolor{poor} 16& \cellcolor{default2} 5& \cellcolor{default2} 0\\
	\cellcolor{default1} BOX2& \cellcolor{default1} True& \cellcolor{default1} 3& \cellcolor{default1} True& \cellcolor{default1} True& \cellcolor{ok} 5.403778E-19& \cellcolor{best} 5.403778E-19& \cellcolor{best} 7& \cellcolor{ok} 8& \cellcolor{default1} 0& \cellcolor{default1} 0\\
	\cellcolor{default2} BOX3& \cellcolor{default2} False& \cellcolor{default2} 3& \cellcolor{default2} True& \cellcolor{default2} True& \cellcolor{ok} 6.871809E-19& \cellcolor{best} 5.382223E-19& \cellcolor{best} 8& \cellcolor{ok} 9& \cellcolor{default2} 0& \cellcolor{default2} 0\\
	\cellcolor{default1} BOXBODLS& \cellcolor{default1} False& \cellcolor{default1} 2& \cellcolor{default1} True& \cellcolor{default1} True& \cellcolor{best} 9.771500E+03& \cellcolor{best} 9.771500E+03& \cellcolor{best} 7& \cellcolor{ok} 13& \cellcolor{default1} 0& \cellcolor{default1} 0\\
	\cellcolor{default2} BQP1VAR& \cellcolor{default2} True& \cellcolor{default2} 1& \cellcolor{default2} True& \cellcolor{default2} True& \cellcolor{ok} 0.000000E+00& \cellcolor{best} -7.443447E-09& \cellcolor{best} 1& \cellcolor{poor} 5& \cellcolor{default2} 0& \cellcolor{default2} 0\\
	\cellcolor{default1} BQPGABIM& \cellcolor{default1} True& \cellcolor{default1} 50& \cellcolor{default1} True& \cellcolor{default1} True& \cellcolor{best} -3.790343E-05& \cellcolor{ok} -3.789187E-05& \cellcolor{best} 2& \cellcolor{poor} 15& \cellcolor{default1} 0& \cellcolor{default1} 0\\
	\cellcolor{default2} BQPGASIM& \cellcolor{default2} True& \cellcolor{default2} 50& \cellcolor{default2} True& \cellcolor{default2} True& \cellcolor{best} -5.519814E-05& \cellcolor{ok} -5.516937E-05& \cellcolor{best} 2& \cellcolor{poor} 15& \cellcolor{default2} 0& \cellcolor{default2} 0\\
	\cellcolor{default1} BQPGAUSS& \cellcolor{default1} True& \cellcolor{default1} 2003& \cellcolor{default1} False& \cellcolor{default1} True& \cellcolor{poor} -1.157101E-02& \cellcolor{best} -3.625779E-01& \cellcolor{best} 20& \cellcolor{ok} 23& \cellcolor{default1} 2& \cellcolor{default1} 0\\
	\cellcolor{default2} BRKMCC& \cellcolor{default2} False& \cellcolor{default2} 2& \cellcolor{default2} True& \cellcolor{default2} True& \cellcolor{best} 1.690427E-01& \cellcolor{ok} 1.690427E-01& \cellcolor{best} 3& \cellcolor{best} 3& \cellcolor{default2} 0& \cellcolor{default2} 0\\
	\cellcolor{default1} BROWNAL& \cellcolor{default1} False& \cellcolor{default1} 200& \cellcolor{default1} True& \cellcolor{default1} True& \cellcolor{best} 5.586575E-25& \cellcolor{ok} 1.091606E-21& \cellcolor{best} 5& \cellcolor{best} 5& \cellcolor{default1} 0& \cellcolor{default1} 0\\
	\cellcolor{default2} BROWNBS& \cellcolor{default2} False& \cellcolor{default2} 2& \cellcolor{default2} True& \cellcolor{default2} True& \cellcolor{best} 0.000000E+00& \cellcolor{best} 0.000000E+00& \cellcolor{ok} 10& \cellcolor{best} 7& \cellcolor{default2} 0& \cellcolor{default2} 0\\
	\cellcolor{default1} BROWNDEN& \cellcolor{default1} False& \cellcolor{default1} 4& \cellcolor{default1} True& \cellcolor{default1} True& \cellcolor{ok} 8.582220E+04& \cellcolor{best} 8.582220E+04& \cellcolor{best} 8& \cellcolor{best} 8& \cellcolor{default1} 0& \cellcolor{default1} 0\\
	\cellcolor{default2} BROWNDENE& \cellcolor{default2} False& \cellcolor{default2} 4& \cellcolor{default2} True& \cellcolor{default2} True& \cellcolor{ok} 9.032343E+02& \cellcolor{best} 9.032343E+02& \cellcolor{best} 1& \cellcolor{best} 1& \cellcolor{default2} 0& \cellcolor{default2} 0\\
	\cellcolor{default1} BROYDN3DLS& \cellcolor{default1} False& \cellcolor{default1} 10& \cellcolor{default1} True& \cellcolor{default1} True& \cellcolor{best} 1.232595E-30& \cellcolor{ok} 2.366583E-30& \cellcolor{best} 6& \cellcolor{best} 6& \cellcolor{default1} 0& \cellcolor{default1} 0\\
	\cellcolor{default2} BROYDN7D& \cellcolor{default2} False& \cellcolor{default2} 5000& \cellcolor{default2} True& \cellcolor{default2} True& \cellcolor{poor} 1.775979E+03& \cellcolor{best} 1.515038E+03& \cellcolor{best} 54& \cellcolor{poor} 121& \cellcolor{default2} 0& \cellcolor{default2} 0\\
	\cellcolor{default1} BROYDNBDLS& \cellcolor{default1} False& \cellcolor{default1} 10& \cellcolor{default1} True& \cellcolor{default1} True& \cellcolor{ok} 1.695783E-17& \cellcolor{best} 7.957487E-18& \cellcolor{best} 11& \cellcolor{best} 11& \cellcolor{default1} 0& \cellcolor{default1} 0\\
	\cellcolor{default2} BRYBND& \cellcolor{default2} False& \cellcolor{default2} 5000& \cellcolor{default2} True& \cellcolor{default2} True& \cellcolor{ok} 2.027851E-20& \cellcolor{best} 4.279822E-21& \cellcolor{ok} 13& \cellcolor{best} 11& \cellcolor{default2} 0& \cellcolor{default2} 0\\
	\cellcolor{default1} CAMEL6& \cellcolor{default1} True& \cellcolor{default1} 2& \cellcolor{default1} True& \cellcolor{default1} True& \cellcolor{best} -1.031628E+00& \cellcolor{ok} -1.031628E+00& \cellcolor{best} 6& \cellcolor{ok} 10& \cellcolor{default1} 0& \cellcolor{default1} 0\\
	\cellcolor{default2} CHAINWOO& \cellcolor{default2} False& \cellcolor{default2} 4000& \cellcolor{default2} False& \cellcolor{default2} True& \cellcolor{poor} 1.574620E+04& \cellcolor{best} 7.933124E+01& \cellcolor{best} 12& \cellcolor{poor} 187& \cellcolor{default2} 2& \cellcolor{default2} 0\\
	\cellcolor{default1} CHEBYQAD& \cellcolor{default1} True& \cellcolor{default1} 100& \cellcolor{default1} True& \cellcolor{default1} True& \cellcolor{poor} 1.196433E-02& \cellcolor{best} 4.877696E-03& \cellcolor{best} 47& \cellcolor{poor} 273& \cellcolor{default1} 0& \cellcolor{default1} 0\\
	\cellcolor{default2} CHENHARK& \cellcolor{default2} True& \cellcolor{default2} 5000& \cellcolor{default2} False& \cellcolor{default2} True& \cellcolor{poor} 9.995000E+02& \cellcolor{best} -2.000002E+00& \cellcolor{best} 0& \cellcolor{poor} 19& \cellcolor{default2} 6& \cellcolor{default2} 0\\
	\cellcolor{default1} CHNROSNB& \cellcolor{default1} False& \cellcolor{default1} 50& \cellcolor{default1} True& \cellcolor{default1} True& \cellcolor{ok} 9.830587E-20& \cellcolor{best} 1.539369E-22& \cellcolor{ok} 50& \cellcolor{best} 42& \cellcolor{default1} 0& \cellcolor{default1} 0\\
	\cellcolor{default2} CHNRSNBM& \cellcolor{default2} False& \cellcolor{default2} 50& \cellcolor{default2} True& \cellcolor{default2} True& \cellcolor{best} 3.598979E-19& \cellcolor{ok} 8.488299E-16& \cellcolor{ok} 56& \cellcolor{best} 52& \cellcolor{default2} 0& \cellcolor{default2} 0\\
	\cellcolor{default1} CHWIRUT1LS& \cellcolor{default1} False& \cellcolor{default1} 3& \cellcolor{default1} False& \cellcolor{default1} True& \cellcolor{poor} 3.710101E+05& \cellcolor{best} 2.384477E+03& \cellcolor{best} 1& \cellcolor{poor} 6& \cellcolor{default1} 5& \cellcolor{default1} 0\\
	\cellcolor{default2} CHWIRUT2LS& \cellcolor{default2} False& \cellcolor{default2} 3& \cellcolor{default2} False& \cellcolor{default2} True& \cellcolor{poor} 1.326249E+07& \cellcolor{best} 5.130480E+02& \cellcolor{best} 1& \cellcolor{poor} 6& \cellcolor{default2} 5& \cellcolor{default2} 0\\
	\cellcolor{default1} CLIFF& \cellcolor{default1} False& \cellcolor{default1} 2& \cellcolor{default1} True& \cellcolor{default1} True& \cellcolor{best} 1.997866E-01& \cellcolor{poor} 2.072380E-01& \cellcolor{ok} 27& \cellcolor{best} 23& \cellcolor{default1} 0& \cellcolor{default1} 0\\
	\cellcolor{default2} CRAGGLVY& \cellcolor{default2} False& \cellcolor{default2} 5000& \cellcolor{default2} True& \cellcolor{default2} True& \cellcolor{ok} 1.688215E+03& \cellcolor{best} 1.688215E+03& \cellcolor{best} 14& \cellcolor{best} 14& \cellcolor{default2} 0& \cellcolor{default2} 0\\
	\cellcolor{default1} CUBE& \cellcolor{default1} False& \cellcolor{default1} 2& \cellcolor{default1} True& \cellcolor{default1} True& \cellcolor{ok} 2.563981E-18& \cellcolor{best} 1.753568E-24& \cellcolor{ok} 28& \cellcolor{best} 27& \cellcolor{default1} 0& \cellcolor{default1} 0\\
	\cellcolor{default2} DANWOODLS& \cellcolor{default2} False& \cellcolor{default2} 2& \cellcolor{default2} True& \cellcolor{default2} True& \cellcolor{ok} 4.317308E-03& \cellcolor{best} 4.317308E-03& \cellcolor{ok} 13& \cellcolor{best} 11& \cellcolor{default2} 0& \cellcolor{default2} 0\\
	\cellcolor{default1} DECONVB& \cellcolor{default1} True& \cellcolor{default1} 63& \cellcolor{default1} True& \cellcolor{default1} False& \cellcolor{ok} 4.032415E-07& \cellcolor{best} 6.952463E-10& \cellcolor{best} 64& \cellcolor{poor} 3000& \cellcolor{default1} 0& \cellcolor{default1} 1\\
	\cellcolor{default2} DECONVU& \cellcolor{default2} True& \cellcolor{default2} 63& \cellcolor{default2} True& \cellcolor{default2} True& \cellcolor{ok} 1.038219E-06& \cellcolor{best} 4.362019E-11& \cellcolor{best} 88& \cellcolor{poor} 182& \cellcolor{default2} 0& \cellcolor{default2} 0\\
	\cellcolor{default1} DENSCHNA& \cellcolor{default1} False& \cellcolor{default1} 2& \cellcolor{default1} True& \cellcolor{default1} True& \cellcolor{ok} 1.102837E-23& \cellcolor{best} 1.102837E-23& \cellcolor{best} 6& \cellcolor{best} 6& \cellcolor{default1} 0& \cellcolor{default1} 0\\
	\cellcolor{default2} DENSCHNB& \cellcolor{default2} False& \cellcolor{default2} 2& \cellcolor{default2} True& \cellcolor{default2} True& \cellcolor{ok} 8.366166E-27& \cellcolor{best} 9.860761E-32& \cellcolor{best} 7& \cellcolor{best} 7& \cellcolor{default2} 0& \cellcolor{default2} 0\\
	\cellcolor{default1} DENSCHNC& \cellcolor{default1} False& \cellcolor{default1} 2& \cellcolor{default1} True& \cellcolor{default1} True& \cellcolor{best} 8.018738E-23& \cellcolor{ok} 2.177679E-20& \cellcolor{ok} 11& \cellcolor{best} 10& \cellcolor{default1} 0& \cellcolor{default1} 0\\
	\cellcolor{default2} DENSCHND& \cellcolor{default2} False& \cellcolor{default2} 3& \cellcolor{default2} True& \cellcolor{default2} True& \cellcolor{best} 5.448220E-09& \cellcolor{poor} 2.221899E-04& \cellcolor{ok} 35& \cellcolor{best} 26& \cellcolor{default2} 0& \cellcolor{default2} 0\\
	\cellcolor{default1} DENSCHNE& \cellcolor{default1} False& \cellcolor{default1} 3& \cellcolor{default1} True& \cellcolor{default1} True& \cellcolor{ok} 2.363081E-17& \cellcolor{best} 1.860553E-17& \cellcolor{poor} 258& \cellcolor{best} 14& \cellcolor{default1} 0& \cellcolor{default1} 0\\
	\cellcolor{default2} DENSCHNF& \cellcolor{default2} False& \cellcolor{default2} 2& \cellcolor{default2} True& \cellcolor{default2} True& \cellcolor{ok} 6.513246E-22& \cellcolor{best} 6.513246E-22& \cellcolor{best} 6& \cellcolor{best} 6& \cellcolor{default2} 0& \cellcolor{default2} 0\\
	\cellcolor{default1} DIXMAANA& \cellcolor{default1} False& \cellcolor{default1} 3000& \cellcolor{default1} True& \cellcolor{default1} True& \cellcolor{best} 1.000000E+00& \cellcolor{best} 1.000000E+00& \cellcolor{best} 6& \cellcolor{ok} 7& \cellcolor{default1} 0& \cellcolor{default1} 0\\
	\cellcolor{default2} DIXMAANB& \cellcolor{default2} False& \cellcolor{default2} 3000& \cellcolor{default2} True& \cellcolor{default2} True& \cellcolor{best} 1.000000E+00& \cellcolor{best} 1.000000E+00& \cellcolor{best} 7& \cellcolor{ok} 11& \cellcolor{default2} 0& \cellcolor{default2} 0\\
	\cellcolor{default1} DIXMAANC& \cellcolor{default1} False& \cellcolor{default1} 3000& \cellcolor{default1} True& \cellcolor{default1} True& \cellcolor{best} 1.000000E+00& \cellcolor{best} 1.000000E+00& \cellcolor{best} 8& \cellcolor{ok} 9& \cellcolor{default1} 0& \cellcolor{default1} 0\\
	\cellcolor{default2} DIXMAAND& \cellcolor{default2} False& \cellcolor{default2} 3000& \cellcolor{default2} True& \cellcolor{default2} True& \cellcolor{best} 1.000000E+00& \cellcolor{best} 1.000000E+00& \cellcolor{best} 9& \cellcolor{best} 9& \cellcolor{default2} 0& \cellcolor{default2} 0\\
	\cellcolor{default1} DIXMAANE& \cellcolor{default1} False& \cellcolor{default1} 3000& \cellcolor{default1} True& \cellcolor{default1} True& \cellcolor{best} 1.000000E+00& \cellcolor{best} 1.000000E+00& \cellcolor{best} 9& \cellcolor{ok} 10& \cellcolor{default1} 0& \cellcolor{default1} 0\\
	\cellcolor{default2} DIXMAANF& \cellcolor{default2} False& \cellcolor{default2} 3000& \cellcolor{default2} True& \cellcolor{default2} True& \cellcolor{best} 1.000000E+00& \cellcolor{best} 1.000000E+00& \cellcolor{ok} 31& \cellcolor{best} 19& \cellcolor{default2} 0& \cellcolor{default2} 0\\
	\cellcolor{default1} DIXMAANG& \cellcolor{default1} False& \cellcolor{default1} 3000& \cellcolor{default1} True& \cellcolor{default1} True& \cellcolor{best} 1.000000E+00& \cellcolor{best} 1.000000E+00& \cellcolor{ok} 32& \cellcolor{best} 16& \cellcolor{default1} 0& \cellcolor{default1} 0\\
	\cellcolor{default2} DIXMAANH& \cellcolor{default2} False& \cellcolor{default2} 3000& \cellcolor{default2} True& \cellcolor{default2} True& \cellcolor{best} 1.000000E+00& \cellcolor{best} 1.000000E+00& \cellcolor{ok} 24& \cellcolor{best} 19& \cellcolor{default2} 0& \cellcolor{default2} 0\\
	\cellcolor{default1} DIXMAANI& \cellcolor{default1} False& \cellcolor{default1} 3000& \cellcolor{default1} False& \cellcolor{default1} True& \cellcolor{poor} 1.166435E+00& \cellcolor{best} 1.000000E+00& \cellcolor{poor} 100& \cellcolor{best} 18& \cellcolor{default1} 2& \cellcolor{default1} 0\\
	\cellcolor{default2} DIXMAANJ& \cellcolor{default2} False& \cellcolor{default2} 3000& \cellcolor{default2} False& \cellcolor{default2} True& \cellcolor{poor} 1.000613E+00& \cellcolor{best} 1.000000E+00& \cellcolor{poor} 83& \cellcolor{best} 20& \cellcolor{default2} 2& \cellcolor{default2} 0\\
	\cellcolor{default1} DIXMAANK& \cellcolor{default1} False& \cellcolor{default1} 3000& \cellcolor{default1} False& \cellcolor{default1} True& \cellcolor{poor} 1.001476E+00& \cellcolor{best} 1.000000E+00& \cellcolor{poor} 83& \cellcolor{best} 24& \cellcolor{default1} 2& \cellcolor{default1} 0\\
	\cellcolor{default2} DIXMAANL& \cellcolor{default2} False& \cellcolor{default2} 3000& \cellcolor{default2} False& \cellcolor{default2} True& \cellcolor{poor} 1.013376E+00& \cellcolor{best} 1.000000E+00& \cellcolor{poor} 83& \cellcolor{best} 27& \cellcolor{default2} 2& \cellcolor{default2} 0\\
	\cellcolor{default1} DIXMAANM& \cellcolor{default1} False& \cellcolor{default1} 3000& \cellcolor{default1} False& \cellcolor{default1} True& \cellcolor{poor} 2.147182E+00& \cellcolor{best} 1.000000E+00& \cellcolor{poor} 100& \cellcolor{best} 11& \cellcolor{default1} 2& \cellcolor{default1} 0\\
	\cellcolor{default2} DIXMAANN& \cellcolor{default2} False& \cellcolor{default2} 3000& \cellcolor{default2} False& \cellcolor{default2} True& \cellcolor{poor} 1.007268E+00& \cellcolor{best} 1.000000E+00& \cellcolor{poor} 81& \cellcolor{best} 25& \cellcolor{default2} 2& \cellcolor{default2} 0\\
	\cellcolor{default1} DIXMAANO& \cellcolor{default1} False& \cellcolor{default1} 3000& \cellcolor{default1} False& \cellcolor{default1} True& \cellcolor{poor} 1.003826E+00& \cellcolor{best} 1.000000E+00& \cellcolor{poor} 82& \cellcolor{best} 25& \cellcolor{default1} 2& \cellcolor{default1} 0\\
	\cellcolor{default2} DIXMAANP& \cellcolor{default2} False& \cellcolor{default2} 3000& \cellcolor{default2} False& \cellcolor{default2} True& \cellcolor{poor} 1.002339E+00& \cellcolor{best} 1.000000E+00& \cellcolor{poor} 81& \cellcolor{best} 28& \cellcolor{default2} 2& \cellcolor{default2} 0\\
	\cellcolor{default1} DJTL& \cellcolor{default1} False& \cellcolor{default1} 2& \cellcolor{default1} True& \cellcolor{default1} True& \cellcolor{ok} -8.951545E+03& \cellcolor{best} -8.951545E+03& \cellcolor{ok} 2541& \cellcolor{best} 1527& \cellcolor{default1} 0& \cellcolor{default1} 0\\
	\cellcolor{default2} DQDRTIC& \cellcolor{default2} False& \cellcolor{default2} 5000& \cellcolor{default2} True& \cellcolor{default2} True& \cellcolor{best} 0.000000E+00& \cellcolor{ok} 5.916457E-29& \cellcolor{best} 1& \cellcolor{best} 1& \cellcolor{default2} 0& \cellcolor{default2} 0\\
	\cellcolor{default1} DQRTIC& \cellcolor{default1} False& \cellcolor{default1} 5000& \cellcolor{default1} False& \cellcolor{default1} True& \cellcolor{poor} 6.240630E+17& \cellcolor{best} 3.935732E+01& \cellcolor{best} 0& \cellcolor{poor} 23& \cellcolor{default1} 2& \cellcolor{default1} 0\\
	\cellcolor{default2} DRCAV1LQ& \cellcolor{default2} True& \cellcolor{default2} 4489& \cellcolor{default2} False& \cellcolor{default2} True& \cellcolor{poor} 1.296927E-04& \cellcolor{best} 5.852343E-15& \cellcolor{best} 22& \cellcolor{poor} 94& \cellcolor{default2} 2& \cellcolor{default2} 0\\
	\cellcolor{default1} DRCAV2LQ& \cellcolor{default1} True& \cellcolor{default1} 4489& \cellcolor{default1} False& \cellcolor{default1} True& \cellcolor{poor} 2.002252E-04& \cellcolor{best} 4.727775E-08& \cellcolor{best} 22& \cellcolor{poor} 169& \cellcolor{default1} 2& \cellcolor{default1} 0\\
	\cellcolor{default2} DRCAV3LQ& \cellcolor{default2} True& \cellcolor{default2} 4489& \cellcolor{default2} False& \cellcolor{default2} True& \cellcolor{poor} 2.181732E-03& \cellcolor{best} 1.371340E-06& \cellcolor{best} 22& \cellcolor{poor} 490& \cellcolor{default2} 2& \cellcolor{default2} 0\\
	\cellcolor{default1} ECKERLE4LS& \cellcolor{default1} False& \cellcolor{default1} 3& \cellcolor{default1} True& \cellcolor{default1} True& \cellcolor{poor} 6.996961E-01& \cellcolor{best} 1.463589E-03& \cellcolor{ok} 37& \cellcolor{best} 36& \cellcolor{default1} 0& \cellcolor{default1} 0\\
	\cellcolor{default2} EDENSCH& \cellcolor{default2} False& \cellcolor{default2} 2000& \cellcolor{default2} True& \cellcolor{default2} True& \cellcolor{ok} 1.200328E+04& \cellcolor{best} 1.200328E+04& \cellcolor{best} 11& \cellcolor{ok} 12& \cellcolor{default2} 0& \cellcolor{default2} 0\\
	\cellcolor{default1} EG1& \cellcolor{default1} True& \cellcolor{default1} 3& \cellcolor{default1} True& \cellcolor{default1} True& \cellcolor{poor} -1.132801E+00& \cellcolor{best} -1.429307E+00& \cellcolor{ok} 12& \cellcolor{best} 7& \cellcolor{default1} 0& \cellcolor{default1} 0\\
	\cellcolor{default2} EG2& \cellcolor{default2} False& \cellcolor{default2} 1000& \cellcolor{default2} True& \cellcolor{default2} True& \cellcolor{ok} -9.989474E+02& \cellcolor{best} -9.989474E+02& \cellcolor{best} 3& \cellcolor{ok} 4& \cellcolor{default2} 0& \cellcolor{default2} 0\\
	\cellcolor{default1} EIGENALS& \cellcolor{default1} False& \cellcolor{default1} 2550& \cellcolor{default1} True& \cellcolor{default1} False& \cellcolor{best} 3.038702E-12& \cellcolor{err} None& \cellcolor{best} 136& \cellcolor{err} None& \cellcolor{default1} 0& \cellcolor{default1} 2\\
	\cellcolor{default2} EIGENBLS& \cellcolor{default2} False& \cellcolor{default2} 2550& \cellcolor{default2} False& \cellcolor{default2} False& \cellcolor{best} 4.958431E-02& \cellcolor{err} None& \cellcolor{best} 135& \cellcolor{err} None& \cellcolor{default2} 2& \cellcolor{default2} 2\\
	\cellcolor{default1} EIGENCLS& \cellcolor{default1} False& \cellcolor{default1} 2652& \cellcolor{default1} False& \cellcolor{default1} False& \cellcolor{best} 1.502568E+03& \cellcolor{err} None& \cellcolor{best} 126& \cellcolor{err} None& \cellcolor{default1} 2& \cellcolor{default1} 2\\
	\cellcolor{default2} ENGVAL1& \cellcolor{default2} False& \cellcolor{default2} 5000& \cellcolor{default2} True& \cellcolor{default2} True& \cellcolor{ok} 5.548668E+03& \cellcolor{best} 5.548668E+03& \cellcolor{best} 8& \cellcolor{best} 8& \cellcolor{default2} 0& \cellcolor{default2} 0\\
	\cellcolor{default1} ENGVAL2& \cellcolor{default1} False& \cellcolor{default1} 3& \cellcolor{default1} True& \cellcolor{default1} True& \cellcolor{best} 2.783412E-22& \cellcolor{ok} 1.700242E-20& \cellcolor{best} 13& \cellcolor{ok} 21& \cellcolor{default1} 0& \cellcolor{default1} 0\\
	\cellcolor{default2} ENSOLS& \cellcolor{default2} False& \cellcolor{default2} 9& \cellcolor{default2} True& \cellcolor{default2} True& \cellcolor{best} 7.885398E+02& \cellcolor{ok} 7.885398E+02& \cellcolor{ok} 9& \cellcolor{best} 7& \cellcolor{default2} 0& \cellcolor{default2} 0\\
	\cellcolor{default1} ERRINROS& \cellcolor{default1} False& \cellcolor{default1} 50& \cellcolor{default1} True& \cellcolor{default1} True& \cellcolor{ok} 4.040449E+01& \cellcolor{best} 4.040449E+01& \cellcolor{ok} 39& \cellcolor{best} 28& \cellcolor{default1} 0& \cellcolor{default1} 0\\
	\cellcolor{default2} ERRINRSM& \cellcolor{default2} False& \cellcolor{default2} 50& \cellcolor{default2} True& \cellcolor{default2} True& \cellcolor{best} 3.851945E+01& \cellcolor{ok} 3.851945E+01& \cellcolor{ok} 54& \cellcolor{best} 40& \cellcolor{default2} 0& \cellcolor{default2} 0\\
	\cellcolor{default1} EXPFIT& \cellcolor{default1} False& \cellcolor{default1} 2& \cellcolor{default1} True& \cellcolor{default1} True& \cellcolor{best} 2.405106E-01& \cellcolor{ok} 2.405106E-01& \cellcolor{best} 6& \cellcolor{ok} 8& \cellcolor{default1} 0& \cellcolor{default1} 0\\
	\cellcolor{default2} EXPLIN& \cellcolor{default2} True& \cellcolor{default2} 1200& \cellcolor{default2} True& \cellcolor{default2} True& \cellcolor{best} -7.192548E+07& \cellcolor{ok} -7.192548E+07& \cellcolor{ok} 72& \cellcolor{best} 57& \cellcolor{default2} 0& \cellcolor{default2} 0\\
	\cellcolor{default1} EXPLIN2& \cellcolor{default1} True& \cellcolor{default1} 1200& \cellcolor{default1} True& \cellcolor{default1} True& \cellcolor{best} -7.199883E+07& \cellcolor{ok} -7.199883E+07& \cellcolor{best} 23& \cellcolor{ok} 25& \cellcolor{default1} 0& \cellcolor{default1} 0\\
	\cellcolor{default2} EXPQUAD& \cellcolor{default2} True& \cellcolor{default2} 1200& \cellcolor{default2} False& \cellcolor{default2} True& \cellcolor{ok} -3.684941E+09& \cellcolor{best} -3.684941E+09& \cellcolor{poor} 138& \cellcolor{best} 32& \cellcolor{default2} 5& \cellcolor{default2} 0\\
	\cellcolor{default1} EXTROSNB& \cellcolor{default1} False& \cellcolor{default1} 1000& \cellcolor{default1} True& \cellcolor{default1} True& \cellcolor{ok} 1.986815E-06& \cellcolor{best} 1.413033E-09& \cellcolor{best} 273& \cellcolor{poor} 2718& \cellcolor{default1} 0& \cellcolor{default1} 0\\
	\cellcolor{default2} FBRAIN2LS& \cellcolor{default2} True& \cellcolor{default2} 4& \cellcolor{default2} True& \cellcolor{default2} True& \cellcolor{best} 3.683882E-01& \cellcolor{ok} 3.683882E-01& \cellcolor{best} 11& \cellcolor{ok} 15& \cellcolor{default2} 0& \cellcolor{default2} 0\\
	\cellcolor{default1} FBRAIN3LS& \cellcolor{default1} False& \cellcolor{default1} 6& \cellcolor{default1} True& \cellcolor{default1} False& \cellcolor{poor} 2.497709E-01& \cellcolor{best} 2.419554E-01& \cellcolor{best} 2004& \cellcolor{ok} 3000& \cellcolor{default1} 0& \cellcolor{default1} 1\\
	\cellcolor{default2} FBRAINLS& \cellcolor{default2} True& \cellcolor{default2} 2& \cellcolor{default2} True& \cellcolor{default2} True& \cellcolor{ok} 4.166029E-01& \cellcolor{best} 4.166029E-01& \cellcolor{best} 7& \cellcolor{ok} 8& \cellcolor{default2} 0& \cellcolor{default2} 0\\
	\cellcolor{default1} FLETBV3M& \cellcolor{default1} False& \cellcolor{default1} 5000& \cellcolor{default1} False& \cellcolor{default1} True& \cellcolor{best} -2.327752E+05& \cellcolor{poor} -2.249399E+05& \cellcolor{best} 26& \cellcolor{poor} 235& \cellcolor{default1} 2& \cellcolor{default1} 0\\
	\cellcolor{default2} FLETCBV2& \cellcolor{default2} False& \cellcolor{default2} 5000& \cellcolor{default2} True& \cellcolor{default2} True& \cellcolor{ok} -5.002682E-01& \cellcolor{best} -5.002863E-01& \cellcolor{best} 1& \cellcolor{best} 1& \cellcolor{default2} 0& \cellcolor{default2} 0\\
	\cellcolor{default1} FLETCBV3& \cellcolor{default1} False& \cellcolor{default1} 5000& \cellcolor{default1} False& \cellcolor{default1} False& \cellcolor{best} -7.310101E+09& \cellcolor{poor} -8.243740E+06& \cellcolor{best} 26& \cellcolor{poor} 3000& \cellcolor{default1} 2& \cellcolor{default1} 1\\
	\cellcolor{default2} FLETCHBV& \cellcolor{default2} False& \cellcolor{default2} 5000& \cellcolor{default2} False& \cellcolor{default2} False& \cellcolor{best} -2.293644E+16& \cellcolor{poor} -8.272254E+14& \cellcolor{best} 26& \cellcolor{poor} 3000& \cellcolor{default2} 2& \cellcolor{default2} 1\\
	\cellcolor{default1} FLETCHCR& \cellcolor{default1} False& \cellcolor{default1} 1000& \cellcolor{default1} True& \cellcolor{default1} True& \cellcolor{ok} 1.326011E-16& \cellcolor{best} 5.294468E-20& \cellcolor{ok} 1568& \cellcolor{best} 1473& \cellcolor{default1} 0& \cellcolor{default1} 0\\
	\cellcolor{default2} FREUROTH& \cellcolor{default2} False& \cellcolor{default2} 5000& \cellcolor{default2} True& \cellcolor{default2} True& \cellcolor{best} 6.081592E+05& \cellcolor{ok} 6.081592E+05& \cellcolor{ok} 12& \cellcolor{best} 8& \cellcolor{default2} 0& \cellcolor{default2} 0\\
	\cellcolor{default1} GAUSS1LS& \cellcolor{default1} False& \cellcolor{default1} 8& \cellcolor{default1} True& \cellcolor{default1} True& \cellcolor{ok} 1.315822E+03& \cellcolor{best} 1.315822E+03& \cellcolor{best} 5& \cellcolor{best} 5& \cellcolor{default1} 0& \cellcolor{default1} 0\\
	\cellcolor{default2} GAUSS2LS& \cellcolor{default2} False& \cellcolor{default2} 8& \cellcolor{default2} True& \cellcolor{default2} True& \cellcolor{ok} 1.247528E+03& \cellcolor{best} 1.247528E+03& \cellcolor{best} 5& \cellcolor{best} 5& \cellcolor{default2} 0& \cellcolor{default2} 0\\
	\cellcolor{default1} GAUSS3LS& \cellcolor{default1} False& \cellcolor{default1} 8& \cellcolor{default1} True& \cellcolor{default1} True& \cellcolor{best} 1.244485E+03& \cellcolor{ok} 1.244485E+03& \cellcolor{best} 8& \cellcolor{ok} 11& \cellcolor{default1} 0& \cellcolor{default1} 0\\
	\cellcolor{default2} GAUSSIAN& \cellcolor{default2} False& \cellcolor{default2} 3& \cellcolor{default2} True& \cellcolor{default2} True& \cellcolor{best} 1.127933E-08& \cellcolor{ok} 1.127933E-08& \cellcolor{best} 2& \cellcolor{best} 2& \cellcolor{default2} 0& \cellcolor{default2} 0\\
	\cellcolor{default1} GBRAINLS& \cellcolor{default1} False& \cellcolor{default1} 2& \cellcolor{default1} True& \cellcolor{default1} True& \cellcolor{best} 2.851586E+01& \cellcolor{ok} 2.851586E+01& \cellcolor{best} 6& \cellcolor{best} 6& \cellcolor{default1} 0& \cellcolor{default1} 0\\
	\cellcolor{default2} GENHUMPS& \cellcolor{default2} False& \cellcolor{default2} 5000& \cellcolor{default2} False& \cellcolor{default2} False& \cellcolor{best} 7.571708E+07& \cellcolor{poor} 8.074689E+07& \cellcolor{best} 23& \cellcolor{poor} 3000& \cellcolor{default2} 2& \cellcolor{default2} 1\\
	\cellcolor{default1} GENROSE& \cellcolor{default1} False& \cellcolor{default1} 500& \cellcolor{default1} True& \cellcolor{default1} True& \cellcolor{best} 1.000000E+00& \cellcolor{best} 1.000000E+00& \cellcolor{best} 344& \cellcolor{ok} 382& \cellcolor{default1} 0& \cellcolor{default1} 0\\
	\cellcolor{default2} GENROSEB& \cellcolor{default2} True& \cellcolor{default2} 500& \cellcolor{default2} True& \cellcolor{default2} True& \cellcolor{best} 1.593945E+03& \cellcolor{ok} 1.593945E+03& \cellcolor{poor} 129& \cellcolor{best} 15& \cellcolor{default2} 0& \cellcolor{default2} 0\\
	\cellcolor{default1} GROWTHLS& \cellcolor{default1} False& \cellcolor{default1} 3& \cellcolor{default1} True& \cellcolor{default1} True& \cellcolor{best} 1.004041E+00& \cellcolor{ok} 1.004041E+00& \cellcolor{ok} 75& \cellcolor{best} 71& \cellcolor{default1} 0& \cellcolor{default1} 0\\
	\cellcolor{default2} GULF& \cellcolor{default2} False& \cellcolor{default2} 3& \cellcolor{default2} True& \cellcolor{default2} True& \cellcolor{ok} 3.645471E-19& \cellcolor{best} 5.933775E-22& \cellcolor{ok} 38& \cellcolor{best} 27& \cellcolor{default2} 0& \cellcolor{default2} 0\\
	\cellcolor{default1} HADAMALS& \cellcolor{default1} True& \cellcolor{default1} 400& \cellcolor{default1} True& \cellcolor{default1} True& \cellcolor{poor} 1.963648E+02& \cellcolor{best} 1.914699E+02& \cellcolor{ok} 187& \cellcolor{best} 162& \cellcolor{default1} 0& \cellcolor{default1} 0\\
	\cellcolor{default2} HAHN1LS& \cellcolor{default2} False& \cellcolor{default2} 7& \cellcolor{default2} False& \cellcolor{default2} True& \cellcolor{poor} 4.727615E+07& \cellcolor{best} 3.338424E+01& \cellcolor{best} 1& \cellcolor{poor} 78& \cellcolor{default2} 5& \cellcolor{default2} 0\\
	\cellcolor{default1} HAIRY& \cellcolor{default1} False& \cellcolor{default1} 2& \cellcolor{default1} True& \cellcolor{default1} True& \cellcolor{best} 2.000000E+01& \cellcolor{best} 2.000000E+01& \cellcolor{best} 33& \cellcolor{ok} 52& \cellcolor{default1} 0& \cellcolor{default1} 0\\
	\cellcolor{default2} HARKERP2& \cellcolor{default2} True& \cellcolor{default2} 1000& \cellcolor{default2} True& \cellcolor{default2} True& \cellcolor{best} -5.000000E-01& \cellcolor{poor} -4.708660E-01& \cellcolor{best} 8& \cellcolor{poor} 23& \cellcolor{default2} 0& \cellcolor{default2} 0\\
	\cellcolor{default1} HART6& \cellcolor{default1} True& \cellcolor{default1} 6& \cellcolor{default1} True& \cellcolor{default1} True& \cellcolor{ok} -3.322887E+00& \cellcolor{best} -3.322887E+00& \cellcolor{best} 5& \cellcolor{ok} 8& \cellcolor{default1} 0& \cellcolor{default1} 0\\
	\cellcolor{default2} HATFLDA& \cellcolor{default2} True& \cellcolor{default2} 4& \cellcolor{default2} True& \cellcolor{default2} True& \cellcolor{best} 4.991961E-22& \cellcolor{ok} 7.237037E-16& \cellcolor{poor} 22& \cellcolor{best} 10& \cellcolor{default2} 0& \cellcolor{default2} 0\\
	\cellcolor{default1} HATFLDB& \cellcolor{default1} True& \cellcolor{default1} 4& \cellcolor{default1} True& \cellcolor{default1} True& \cellcolor{best} 5.572809E-03& \cellcolor{ok} 5.572812E-03& \cellcolor{ok} 19& \cellcolor{best} 10& \cellcolor{default1} 0& \cellcolor{default1} 0\\
	\cellcolor{default2} HATFLDC& \cellcolor{default2} True& \cellcolor{default2} 25& \cellcolor{default2} True& \cellcolor{default2} True& \cellcolor{best} 3.418825E-27& \cellcolor{ok} 2.921509E-17& \cellcolor{best} 5& \cellcolor{best} 5& \cellcolor{default2} 0& \cellcolor{default2} 0\\
	\cellcolor{default1} HATFLDD& \cellcolor{default1} False& \cellcolor{default1} 3& \cellcolor{default1} True& \cellcolor{default1} True& \cellcolor{ok} 6.615134E-08& \cellcolor{best} 6.615114E-08& \cellcolor{best} 19& \cellcolor{ok} 21& \cellcolor{default1} 0& \cellcolor{default1} 0\\
	\cellcolor{default2} HATFLDE& \cellcolor{default2} False& \cellcolor{default2} 3& \cellcolor{default2} True& \cellcolor{default2} True& \cellcolor{best} 5.120000E-07& \cellcolor{ok} 5.120377E-07& \cellcolor{best} 18& \cellcolor{ok} 20& \cellcolor{default2} 0& \cellcolor{default2} 0\\
	\cellcolor{default1} HATFLDFL& \cellcolor{default1} False& \cellcolor{default1} 3& \cellcolor{default1} True& \cellcolor{default1} True& \cellcolor{ok} 6.626283E-05& \cellcolor{best} 6.016514E-05& \cellcolor{best} 4& \cellcolor{poor} 1233& \cellcolor{default1} 0& \cellcolor{default1} 0\\
	\cellcolor{default2} HEART6LS& \cellcolor{default2} False& \cellcolor{default2} 6& \cellcolor{default2} True& \cellcolor{default2} True& \cellcolor{ok} 7.175581E-18& \cellcolor{best} 9.127834E-23& \cellcolor{best} 373& \cellcolor{poor} 878& \cellcolor{default2} 0& \cellcolor{default2} 0\\
	\cellcolor{default1} HEART8LS& \cellcolor{default1} False& \cellcolor{default1} 8& \cellcolor{default1} True& \cellcolor{default1} True& \cellcolor{ok} 7.013926E-19& \cellcolor{best} 6.309963E-29& \cellcolor{poor} 475& \cellcolor{best} 106& \cellcolor{default1} 0& \cellcolor{default1} 0\\
	\cellcolor{default2} HELIX& \cellcolor{default2} False& \cellcolor{default2} 3& \cellcolor{default2} True& \cellcolor{default2} True& \cellcolor{ok} 7.912620E-17& \cellcolor{best} 6.057699E-25& \cellcolor{ok} 14& \cellcolor{best} 13& \cellcolor{default2} 0& \cellcolor{default2} 0\\
	\cellcolor{default1} HIELOW& \cellcolor{default1} False& \cellcolor{default1} 3& \cellcolor{default1} True& \cellcolor{default1} True& \cellcolor{best} -4.789078E+06& \cellcolor{poor} 8.741654E+02& \cellcolor{best} 2& \cellcolor{poor} 8& \cellcolor{default1} 0& \cellcolor{default1} 0\\
	\cellcolor{default2} HILBERTA& \cellcolor{default2} False& \cellcolor{default2} 2& \cellcolor{default2} True& \cellcolor{default2} True& \cellcolor{ok} 2.958228E-31& \cellcolor{best} 1.314768E-31& \cellcolor{best} 1& \cellcolor{best} 1& \cellcolor{default2} 0& \cellcolor{default2} 0\\
	\cellcolor{default1} HILBERTB& \cellcolor{default1} False& \cellcolor{default1} 10& \cellcolor{default1} True& \cellcolor{default1} True& \cellcolor{ok} 2.370815E-29& \cellcolor{best} 7.067769E-30& \cellcolor{best} 1& \cellcolor{best} 1& \cellcolor{default1} 0& \cellcolor{default1} 0\\
	\cellcolor{default2} HIMMELBB& \cellcolor{default2} False& \cellcolor{default2} 2& \cellcolor{default2} True& \cellcolor{default2} True& \cellcolor{ok} 7.334173E-17& \cellcolor{best} 1.401396E-17& \cellcolor{best} 5& \cellcolor{poor} 18& \cellcolor{default2} 0& \cellcolor{default2} 0\\
	\cellcolor{default1} HIMMELBF& \cellcolor{default1} False& \cellcolor{default1} 4& \cellcolor{default1} False& \cellcolor{default1} True& \cellcolor{ok} 3.185790E+02& \cellcolor{best} 3.185717E+02& \cellcolor{poor} 5001& \cellcolor{best} 75& \cellcolor{default1} 1& \cellcolor{default1} 0\\
	\cellcolor{default2} HIMMELBG& \cellcolor{default2} False& \cellcolor{default2} 2& \cellcolor{default2} True& \cellcolor{default2} True& \cellcolor{best} 8.900056E-27& \cellcolor{ok} 3.633000E-22& \cellcolor{best} 5& \cellcolor{ok} 6& \cellcolor{default2} 0& \cellcolor{default2} 0\\
	\cellcolor{default1} HIMMELBH& \cellcolor{default1} False& \cellcolor{default1} 2& \cellcolor{default1} True& \cellcolor{default1} True& \cellcolor{best} -1.000000E+00& \cellcolor{best} -1.000000E+00& \cellcolor{ok} 6& \cellcolor{best} 4& \cellcolor{default1} 0& \cellcolor{default1} 0\\
	\cellcolor{default2} HIMMELP1& \cellcolor{default2} True& \cellcolor{default2} 2& \cellcolor{default2} True& \cellcolor{default2} True& \cellcolor{poor} -2.389742E+01& \cellcolor{best} -6.205394E+01& \cellcolor{poor} 56& \cellcolor{best} 11& \cellcolor{default2} 0& \cellcolor{default2} 0\\
	\cellcolor{default1} HOLMES& \cellcolor{default1} True& \cellcolor{default1} 180& \cellcolor{default1} True& \cellcolor{default1} True& \cellcolor{ok} 1.248150E+03& \cellcolor{best} 1.248150E+03& \cellcolor{ok} 20& \cellcolor{best} 12& \cellcolor{default1} 0& \cellcolor{default1} 0\\
	\cellcolor{default2} HS1& \cellcolor{default2} True& \cellcolor{default2} 2& \cellcolor{default2} True& \cellcolor{default2} True& \cellcolor{best} 1.965084E-23& \cellcolor{ok} 5.894626E-16& \cellcolor{ok} 26& \cellcolor{best} 25& \cellcolor{default2} 0& \cellcolor{default2} 0\\
	\cellcolor{default1} HS110& \cellcolor{default1} True& \cellcolor{default1} 10& \cellcolor{default1} True& \cellcolor{default1} True& \cellcolor{ok} -4.577848E+01& \cellcolor{best} -4.577848E+01& \cellcolor{best} 6& \cellcolor{best} 6& \cellcolor{default1} 0& \cellcolor{default1} 0\\
	\cellcolor{default2} HS2& \cellcolor{default2} True& \cellcolor{default2} 2& \cellcolor{default2} True& \cellcolor{default2} True& \cellcolor{ok} 4.941229E+00& \cellcolor{best} 4.941229E+00& \cellcolor{best} 8& \cellcolor{ok} 11& \cellcolor{default2} 0& \cellcolor{default2} 0\\
	\cellcolor{default1} HS25& \cellcolor{default1} True& \cellcolor{default1} 3& \cellcolor{default1} True& \cellcolor{default1} True& \cellcolor{poor} 3.283500E+01& \cellcolor{best} 1.034604E-15& \cellcolor{best} 1& \cellcolor{poor} 36& \cellcolor{default1} 0& \cellcolor{default1} 0\\
	\cellcolor{default2} HS3& \cellcolor{default2} True& \cellcolor{default2} 2& \cellcolor{default2} True& \cellcolor{default2} True& \cellcolor{ok} 2.174944E-08& \cellcolor{best} -7.494096E-09& \cellcolor{poor} 31& \cellcolor{best} 4& \cellcolor{default2} 0& \cellcolor{default2} 0\\
	\cellcolor{default1} HS38& \cellcolor{default1} True& \cellcolor{default1} 4& \cellcolor{default1} True& \cellcolor{default1} True& \cellcolor{ok} 1.533400E-18& \cellcolor{best} 2.761247E-19& \cellcolor{ok} 41& \cellcolor{best} 40& \cellcolor{default1} 0& \cellcolor{default1} 0\\
	\cellcolor{default2} HS3MOD& \cellcolor{default2} True& \cellcolor{default2} 2& \cellcolor{default2} True& \cellcolor{default2} True& \cellcolor{ok} 4.271062E-10& \cellcolor{best} -7.494096E-09& \cellcolor{poor} 12& \cellcolor{best} 5& \cellcolor{default2} 0& \cellcolor{default2} 0\\
	\cellcolor{default1} HS4& \cellcolor{default1} True& \cellcolor{default1} 2& \cellcolor{default1} True& \cellcolor{default1} True& \cellcolor{best} 2.666667E+00& \cellcolor{ok} 2.666667E+00& \cellcolor{best} 1& \cellcolor{poor} 5& \cellcolor{default1} 0& \cellcolor{default1} 0\\
	\cellcolor{default2} HS45& \cellcolor{default2} True& \cellcolor{default2} 5& \cellcolor{default2} True& \cellcolor{default2} True& \cellcolor{ok} 1.000000E+00& \cellcolor{best} 1.000000E+00& \cellcolor{best} 3& \cellcolor{poor} 7& \cellcolor{default2} 0& \cellcolor{default2} 0\\
	\cellcolor{default1} HS5& \cellcolor{default1} True& \cellcolor{default1} 2& \cellcolor{default1} True& \cellcolor{default1} True& \cellcolor{ok} -1.913223E+00& \cellcolor{best} -1.913223E+00& \cellcolor{best} 5& \cellcolor{ok} 8& \cellcolor{default1} 0& \cellcolor{default1} 0\\
	\cellcolor{default2} HUMPS& \cellcolor{default2} False& \cellcolor{default2} 2& \cellcolor{default2} True& \cellcolor{default2} True& \cellcolor{ok} 2.098890E-14& \cellcolor{best} 1.879074E-23& \cellcolor{best} 87& \cellcolor{poor} 323& \cellcolor{default2} 0& \cellcolor{default2} 0\\
	\cellcolor{default1} HYDC20LS& \cellcolor{default1} False& \cellcolor{default1} 99& \cellcolor{default1} False& \cellcolor{default1} True& \cellcolor{best} 7.556457E-03& \cellcolor{poor} 7.695631E-02& \cellcolor{poor} 5001& \cellcolor{best} 775& \cellcolor{default1} 1& \cellcolor{default1} 0\\
	\cellcolor{default2} INDEF& \cellcolor{default2} False& \cellcolor{default2} 5000& \cellcolor{default2} False& \cellcolor{default2} True& \cellcolor{poor} -5.243019E+09& \cellcolor{best} -2.777566E+20& \cellcolor{best} 22& \cellcolor{poor} 125& \cellcolor{default2} 2& \cellcolor{default2} 0\\
	\cellcolor{default1} INTEQNELS& \cellcolor{default1} False& \cellcolor{default1} 12& \cellcolor{default1} True& \cellcolor{default1} True& \cellcolor{best} 3.994096E-22& \cellcolor{ok} 3.994096E-22& \cellcolor{best} 3& \cellcolor{best} 3& \cellcolor{default1} 0& \cellcolor{default1} 0\\
	\cellcolor{default2} JENSMP& \cellcolor{default2} False& \cellcolor{default2} 2& \cellcolor{default2} True& \cellcolor{default2} True& \cellcolor{best} 1.243622E+02& \cellcolor{ok} 1.243622E+02& \cellcolor{ok} 10& \cellcolor{best} 9& \cellcolor{default2} 0& \cellcolor{default2} 0\\
	\cellcolor{default1} JIMACK& \cellcolor{default1} False& \cellcolor{default1} 3549& \cellcolor{default1} False& \cellcolor{default1} True& \cellcolor{poor} 8.841238E-01& \cellcolor{best} 8.667933E-01& \cellcolor{poor} 47& \cellcolor{best} 18& \cellcolor{default1} 2& \cellcolor{default1} 0\\
	\cellcolor{default2} KIRBY2LS& \cellcolor{default2} False& \cellcolor{default2} 5& \cellcolor{default2} False& \cellcolor{default2} True& \cellcolor{poor} 1.869869E+06& \cellcolor{best} 3.905074E+00& \cellcolor{best} 1& \cellcolor{poor} 11& \cellcolor{default2} 5& \cellcolor{default2} 0\\
	\cellcolor{default1} KOEBHELB& \cellcolor{default1} True& \cellcolor{default1} 3& \cellcolor{default1} True& \cellcolor{default1} True& \cellcolor{best} 7.751635E+01& \cellcolor{ok} 7.751635E+01& \cellcolor{best} 77& \cellcolor{poor} 345& \cellcolor{default1} 0& \cellcolor{default1} 0\\
	\cellcolor{default2} KOWOSB& \cellcolor{default2} False& \cellcolor{default2} 4& \cellcolor{default2} True& \cellcolor{default2} True& \cellcolor{best} 3.078009E-04& \cellcolor{best} 3.078009E-04& \cellcolor{poor} 62& \cellcolor{best} 8& \cellcolor{default2} 0& \cellcolor{default2} 0\\
	\cellcolor{default1} LANCZOS1LS& \cellcolor{default1} False& \cellcolor{default1} 6& \cellcolor{default1} True& \cellcolor{default1} True& \cellcolor{ok} 1.637700E-05& \cellcolor{best} 4.285594E-17& \cellcolor{poor} 660& \cellcolor{best} 169& \cellcolor{default1} 0& \cellcolor{default1} 0\\
	\cellcolor{default2} LANCZOS2LS& \cellcolor{default2} False& \cellcolor{default2} 6& \cellcolor{default2} True& \cellcolor{default2} True& \cellcolor{ok} 1.669310E-05& \cellcolor{best} 2.229943E-11& \cellcolor{poor} 669& \cellcolor{best} 102& \cellcolor{default2} 0& \cellcolor{default2} 0\\
	\cellcolor{default1} LANCZOS3LS& \cellcolor{default1} False& \cellcolor{default1} 6& \cellcolor{default1} True& \cellcolor{default1} True& \cellcolor{ok} 1.633548E-05& \cellcolor{best} 1.611719E-08& \cellcolor{poor} 679& \cellcolor{best} 159& \cellcolor{default1} 0& \cellcolor{default1} 0\\
	\cellcolor{default2} LIARWHD& \cellcolor{default2} False& \cellcolor{default2} 5000& \cellcolor{default2} True& \cellcolor{default2} True& \cellcolor{ok} 6.380775E-22& \cellcolor{best} 6.380775E-22& \cellcolor{best} 12& \cellcolor{best} 12& \cellcolor{default2} 0& \cellcolor{default2} 0\\
	\cellcolor{default1} LINVERSE& \cellcolor{default1} True& \cellcolor{default1} 1999& \cellcolor{default1} True& \cellcolor{default1} True& \cellcolor{poor} 6.820000E+02& \cellcolor{best} 6.810000E+02& \cellcolor{best} 128& \cellcolor{poor} 745& \cellcolor{default1} 0& \cellcolor{default1} 0\\
	\cellcolor{default2} LOGHAIRY& \cellcolor{default2} False& \cellcolor{default2} 2& \cellcolor{default2} True& \cellcolor{default2} True& \cellcolor{poor} 6.102268E+00& \cellcolor{best} 1.823216E-01& \cellcolor{best} 565& \cellcolor{poor} 2245& \cellcolor{default2} 0& \cellcolor{default2} 0\\
	\cellcolor{default1} LOGROS& \cellcolor{default1} True& \cellcolor{default1} 2& \cellcolor{default1} True& \cellcolor{default1} True& \cellcolor{best} 0.000000E+00& \cellcolor{best} 0.000000E+00& \cellcolor{best} 53& \cellcolor{ok} 65& \cellcolor{default1} 0& \cellcolor{default1} 0\\
	\cellcolor{default2} LSC1LS& \cellcolor{default2} False& \cellcolor{default2} 3& \cellcolor{default2} True& \cellcolor{default2} True& \cellcolor{ok} 7.711852E+00& \cellcolor{best} 7.711852E+00& \cellcolor{poor} 67& \cellcolor{best} 16& \cellcolor{default2} 0& \cellcolor{default2} 0\\
	\cellcolor{default1} LSC2LS& \cellcolor{default1} False& \cellcolor{default1} 3& \cellcolor{default1} False& \cellcolor{default1} True& \cellcolor{poor} 1.403637E+01& \cellcolor{best} 1.333415E+01& \cellcolor{poor} 5001& \cellcolor{best} 44& \cellcolor{default1} 1& \cellcolor{default1} 0\\
	\cellcolor{default2} LUKSAN11LS& \cellcolor{default2} False& \cellcolor{default2} 100& \cellcolor{default2} True& \cellcolor{default2} True& \cellcolor{ok} 8.530675E-18& \cellcolor{best} 1.949090E-27& \cellcolor{ok} 344& \cellcolor{best} 333& \cellcolor{default2} 0& \cellcolor{default2} 0\\
	\cellcolor{default1} LUKSAN12LS& \cellcolor{default1} False& \cellcolor{default1} 98& \cellcolor{default1} True& \cellcolor{default1} True& \cellcolor{best} 4.228836E+03& \cellcolor{poor} 4.292197E+03& \cellcolor{ok} 33& \cellcolor{best} 25& \cellcolor{default1} 0& \cellcolor{default1} 0\\
	\cellcolor{default2} LUKSAN13LS& \cellcolor{default2} False& \cellcolor{default2} 98& \cellcolor{default2} True& \cellcolor{default2} True& \cellcolor{best} 2.518886E+04& \cellcolor{ok} 2.518886E+04& \cellcolor{best} 19& \cellcolor{best} 19& \cellcolor{default2} 0& \cellcolor{default2} 0\\
	\cellcolor{default1} LUKSAN14LS& \cellcolor{default1} False& \cellcolor{default1} 98& \cellcolor{default1} True& \cellcolor{default1} True& \cellcolor{ok} 1.239235E+02& \cellcolor{best} 1.239235E+02& \cellcolor{best} 11& \cellcolor{best} 11& \cellcolor{default1} 0& \cellcolor{default1} 0\\
	\cellcolor{default2} LUKSAN15LS& \cellcolor{default2} False& \cellcolor{default2} 100& \cellcolor{default2} True& \cellcolor{default2} True& \cellcolor{poor} 4.168876E+02& \cellcolor{best} 3.569697E+00& \cellcolor{best} 8& \cellcolor{ok} 9& \cellcolor{default2} 0& \cellcolor{default2} 0\\
	\cellcolor{default1} LUKSAN16LS& \cellcolor{default1} False& \cellcolor{default1} 100& \cellcolor{default1} True& \cellcolor{default1} True& \cellcolor{best} 3.569697E+00& \cellcolor{ok} 3.569697E+00& \cellcolor{ok} 7& \cellcolor{best} 6& \cellcolor{default1} 0& \cellcolor{default1} 0\\
	\cellcolor{default2} LUKSAN17LS& \cellcolor{default2} False& \cellcolor{default2} 100& \cellcolor{default2} True& \cellcolor{default2} True& \cellcolor{best} 4.931613E-01& \cellcolor{ok} 4.931613E-01& \cellcolor{ok} 27& \cellcolor{best} 16& \cellcolor{default2} 0& \cellcolor{default2} 0\\
	\cellcolor{default1} LUKSAN21LS& \cellcolor{default1} False& \cellcolor{default1} 100& \cellcolor{default1} True& \cellcolor{default1} True& \cellcolor{best} 3.729845E-20& \cellcolor{ok} 2.610386E-17& \cellcolor{poor} 289& \cellcolor{best} 11& \cellcolor{default1} 0& \cellcolor{default1} 0\\
	\cellcolor{default2} LUKSAN22LS& \cellcolor{default2} False& \cellcolor{default2} 100& \cellcolor{default2} True& \cellcolor{default2} True& \cellcolor{best} 8.689405E+02& \cellcolor{ok} 8.689405E+02& \cellcolor{ok} 22& \cellcolor{best} 16& \cellcolor{default2} 0& \cellcolor{default2} 0\\
	\cellcolor{default1} MANCINO& \cellcolor{default1} False& \cellcolor{default1} 100& \cellcolor{default1} True& \cellcolor{default1} True& \cellcolor{best} 1.239764E-21& \cellcolor{ok} 1.433993E-21& \cellcolor{best} 15& \cellcolor{ok} 18& \cellcolor{default1} 0& \cellcolor{default1} 0\\
	\cellcolor{default2} MARATOSB& \cellcolor{default2} False& \cellcolor{default2} 2& \cellcolor{default2} True& \cellcolor{default2} True& \cellcolor{best} -1.000000E+00& \cellcolor{ok} -1.000000E+00& \cellcolor{ok} 734& \cellcolor{best} 670& \cellcolor{default2} 0& \cellcolor{default2} 0\\
	\cellcolor{default1} MAXLIKA& \cellcolor{default1} True& \cellcolor{default1} 8& \cellcolor{default1} True& \cellcolor{default1} True& \cellcolor{ok} 1.136307E+03& \cellcolor{best} 1.136307E+03& \cellcolor{poor} 315& \cellcolor{best} 28& \cellcolor{default1} 0& \cellcolor{default1} 0\\
	\cellcolor{default2} MCCORMCK& \cellcolor{default2} True& \cellcolor{default2} 5000& \cellcolor{default2} True& \cellcolor{default2} True& \cellcolor{ok} -4.566581E+03& \cellcolor{best} -4.566581E+03& \cellcolor{best} 6& \cellcolor{ok} 7& \cellcolor{default2} 0& \cellcolor{default2} 0\\
	\cellcolor{default1} MDHOLE& \cellcolor{default1} True& \cellcolor{default1} 2& \cellcolor{default1} True& \cellcolor{default1} True& \cellcolor{ok} 2.889774E-17& \cellcolor{best} -2.261664E-09& \cellcolor{best} 38& \cellcolor{ok} 43& \cellcolor{default1} 0& \cellcolor{default1} 0\\
	\cellcolor{default2} MEXHAT& \cellcolor{default2} False& \cellcolor{default2} 2& \cellcolor{default2} True& \cellcolor{default2} True& \cellcolor{best} -4.001000E-02& \cellcolor{best} -4.001000E-02& \cellcolor{ok} 28& \cellcolor{best} 26& \cellcolor{default2} 0& \cellcolor{default2} 0\\
	\cellcolor{default1} MEYER3& \cellcolor{default1} False& \cellcolor{default1} 3& \cellcolor{default1} False& \cellcolor{default1} True& \cellcolor{poor} 1.761481E+09& \cellcolor{best} 8.794586E+01& \cellcolor{best} 1& \cellcolor{poor} 195& \cellcolor{default1} 5& \cellcolor{default1} 0\\
	\cellcolor{default2} MGH09LS& \cellcolor{default2} False& \cellcolor{default2} 4& \cellcolor{default2} False& \cellcolor{default2} True& \cellcolor{poor} 3.514219E-03& \cellcolor{best} 3.075056E-04& \cellcolor{poor} 5001& \cellcolor{best} 71& \cellcolor{default2} 1& \cellcolor{default2} 0\\
	\cellcolor{default1} MGH10LS& \cellcolor{default1} False& \cellcolor{default1} 3& \cellcolor{default1} False& \cellcolor{default1} True& \cellcolor{poor} 1.417863E+09& \cellcolor{best} 8.794586E+01& \cellcolor{poor} 5001& \cellcolor{best} 1724& \cellcolor{default1} 1& \cellcolor{default1} 0\\
	\cellcolor{default2} MGH17LS& \cellcolor{default2} False& \cellcolor{default2} 5& \cellcolor{default2} True& \cellcolor{default2} True& \cellcolor{poor} 1.022432E+00& \cellcolor{best} 5.464895E-05& \cellcolor{best} 11& \cellcolor{poor} 324& \cellcolor{default2} 0& \cellcolor{default2} 0\\
	\cellcolor{default1} MINSURF& \cellcolor{default1} True& \cellcolor{default1} 64& \cellcolor{default1} True& \cellcolor{default1} True& \cellcolor{ok} 1.000000E+00& \cellcolor{best} 1.000000E+00& \cellcolor{poor} 31& \cellcolor{best} 4& \cellcolor{default1} 0& \cellcolor{default1} 0\\
	\cellcolor{default2} MISRA1ALS& \cellcolor{default2} False& \cellcolor{default2} 2& \cellcolor{default2} True& \cellcolor{default2} True& \cellcolor{best} 1.245514E-01& \cellcolor{ok} 1.245514E-01& \cellcolor{best} 37& \cellcolor{ok} 40& \cellcolor{default2} 0& \cellcolor{default2} 0\\
	\cellcolor{default1} MISRA1BLS& \cellcolor{default1} False& \cellcolor{default1} 2& \cellcolor{default1} True& \cellcolor{default1} True& \cellcolor{ok} 7.546468E-02& \cellcolor{best} 7.546468E-02& \cellcolor{best} 28& \cellcolor{ok} 34& \cellcolor{default1} 0& \cellcolor{default1} 0\\
	\cellcolor{default2} MISRA1CLS& \cellcolor{default2} False& \cellcolor{default2} 2& \cellcolor{default2} True& \cellcolor{default2} True& \cellcolor{best} 4.096684E-02& \cellcolor{ok} 4.096684E-02& \cellcolor{ok} 18& \cellcolor{best} 14& \cellcolor{default2} 0& \cellcolor{default2} 0\\
	\cellcolor{default1} MISRA1DLS& \cellcolor{default1} False& \cellcolor{default1} 2& \cellcolor{default1} True& \cellcolor{default1} True& \cellcolor{best} 5.641930E-02& \cellcolor{ok} 5.641930E-02& \cellcolor{best} 24& \cellcolor{ok} 30& \cellcolor{default1} 0& \cellcolor{default1} 0\\
	\cellcolor{default2} MOREBV& \cellcolor{default2} False& \cellcolor{default2} 5000& \cellcolor{default2} True& \cellcolor{default2} True& \cellcolor{ok} 7.704646E-09& \cellcolor{best} 5.831055E-15& \cellcolor{poor} 4& \cellcolor{best} 1& \cellcolor{default2} 0& \cellcolor{default2} 0\\
	\cellcolor{default1} MSQRTALS& \cellcolor{default1} False& \cellcolor{default1} 1024& \cellcolor{default1} True& \cellcolor{default1} True& \cellcolor{best} 4.968121E-18& \cellcolor{ok} 4.223720E-16& \cellcolor{poor} 84& \cellcolor{best} 24& \cellcolor{default1} 0& \cellcolor{default1} 0\\
	\cellcolor{default2} MSQRTBLS& \cellcolor{default2} False& \cellcolor{default2} 1024& \cellcolor{default2} True& \cellcolor{default2} True& \cellcolor{ok} 1.909138E-18& \cellcolor{best} 1.365103E-21& \cellcolor{ok} 44& \cellcolor{best} 24& \cellcolor{default2} 0& \cellcolor{default2} 0\\
	\cellcolor{default1} NCB20B& \cellcolor{default1} False& \cellcolor{default1} 5000& \cellcolor{default1} True& \cellcolor{default1} True& \cellcolor{best} 7.351301E+03& \cellcolor{ok} 7.351301E+03& \cellcolor{best} 9& \cellcolor{ok} 16& \cellcolor{default1} 0& \cellcolor{default1} 0\\
	\cellcolor{default2} NELSONLS& \cellcolor{default2} False& \cellcolor{default2} 3& \cellcolor{default2} True& \cellcolor{default2} True& \cellcolor{ok} 3.797683E+00& \cellcolor{best} 3.797683E+00& \cellcolor{best} 68& \cellcolor{best} 68& \cellcolor{default2} 0& \cellcolor{default2} 0\\
	\cellcolor{default1} NONCVXU2& \cellcolor{default1} False& \cellcolor{default1} 5000& \cellcolor{default1} False& \cellcolor{default1} False& \cellcolor{best} 1.317778E+08& \cellcolor{err} None& \cellcolor{best} 18& \cellcolor{err} None& \cellcolor{default1} 2& \cellcolor{default1} 2\\
	\cellcolor{default2} NONCVXUN& \cellcolor{default2} False& \cellcolor{default2} 5000& \cellcolor{default2} False& \cellcolor{default2} True& \cellcolor{poor} 1.561126E+08& \cellcolor{best} 1.159976E+04& \cellcolor{best} 19& \cellcolor{poor} 2490& \cellcolor{default2} 2& \cellcolor{default2} 0\\
	\cellcolor{default1} NONDIA& \cellcolor{default1} False& \cellcolor{default1} 5000& \cellcolor{default1} False& \cellcolor{default1} True& \cellcolor{ok} 4.897136E-06& \cellcolor{best} 5.226135E-13& \cellcolor{poor} 21& \cellcolor{best} 7& \cellcolor{default1} 2& \cellcolor{default1} 0\\
	\cellcolor{default2} NONDQUAR& \cellcolor{default2} False& \cellcolor{default2} 5000& \cellcolor{default2} False& \cellcolor{default2} True& \cellcolor{poor} 4.775167E-03& \cellcolor{best} 2.069477E-10& \cellcolor{best} 18& \cellcolor{ok} 19& \cellcolor{default2} 2& \cellcolor{default2} 0\\
	\cellcolor{default1} NONMSQRT& \cellcolor{default1} False& \cellcolor{default1} 4900& \cellcolor{default1} False& \cellcolor{default1} False& \cellcolor{best} 1.267592E+03& \cellcolor{err} None& \cellcolor{best} 24& \cellcolor{err} None& \cellcolor{default1} 2& \cellcolor{default1} 2\\
	\cellcolor{default2} NONSCOMP& \cellcolor{default2} True& \cellcolor{default2} 5000& \cellcolor{default2} True& \cellcolor{default2} True& \cellcolor{best} 2.198811E-12& \cellcolor{ok} 1.233203E-05& \cellcolor{best} 12& \cellcolor{ok} 21& \cellcolor{default2} 0& \cellcolor{default2} 0\\
	\cellcolor{default1} OSBORNEA& \cellcolor{default1} False& \cellcolor{default1} 5& \cellcolor{default1} True& \cellcolor{default1} True& \cellcolor{poor} 4.902382E-02& \cellcolor{best} 5.464895E-05& \cellcolor{ok} 126& \cellcolor{best} 64& \cellcolor{default1} 0& \cellcolor{default1} 0\\
	\cellcolor{default2} OSBORNEB& \cellcolor{default2} False& \cellcolor{default2} 11& \cellcolor{default2} True& \cellcolor{default2} True& \cellcolor{best} 4.013774E-02& \cellcolor{ok} 4.013774E-02& \cellcolor{ok} 31& \cellcolor{best} 19& \cellcolor{default2} 0& \cellcolor{default2} 0\\
	\cellcolor{default1} OSCIPATH& \cellcolor{default1} False& \cellcolor{default1} 10& \cellcolor{default1} True& \cellcolor{default1} False& \cellcolor{poor} 9.999667E-01& \cellcolor{best} 9.994943E-01& \cellcolor{best} 2& \cellcolor{poor} 3000& \cellcolor{default1} 0& \cellcolor{default1} 1\\
	\cellcolor{default2} OSLBQP& \cellcolor{default2} True& \cellcolor{default2} 8& \cellcolor{default2} True& \cellcolor{default2} True& \cellcolor{best} 6.250000E+00& \cellcolor{best} 6.250000E+00& \cellcolor{best} 1& \cellcolor{poor} 14& \cellcolor{default2} 0& \cellcolor{default2} 0\\
	\cellcolor{default1} PALMER1& \cellcolor{default1} True& \cellcolor{default1} 4& \cellcolor{default1} True& \cellcolor{default1} True& \cellcolor{ok} 1.175460E+04& \cellcolor{best} 1.175460E+04& \cellcolor{best} 8& \cellcolor{poor} 392& \cellcolor{default1} 0& \cellcolor{default1} 0\\
	\cellcolor{default2} PALMER1A& \cellcolor{default2} True& \cellcolor{default2} 6& \cellcolor{default2} True& \cellcolor{default2} True& \cellcolor{best} 8.988306E-02& \cellcolor{ok} 8.988306E-02& \cellcolor{best} 35& \cellcolor{ok} 45& \cellcolor{default2} 0& \cellcolor{default2} 0\\
	\cellcolor{default1} PALMER1B& \cellcolor{default1} True& \cellcolor{default1} 4& \cellcolor{default1} True& \cellcolor{default1} True& \cellcolor{ok} 3.447349E+00& \cellcolor{best} 3.447349E+00& \cellcolor{best} 15& \cellcolor{ok} 20& \cellcolor{default1} 0& \cellcolor{default1} 0\\
	\cellcolor{default2} PALMER1C& \cellcolor{default2} False& \cellcolor{default2} 8& \cellcolor{default2} True& \cellcolor{default2} True& \cellcolor{best} 9.760505E-02& \cellcolor{ok} 9.760505E-02& \cellcolor{ok} 2& \cellcolor{best} 1& \cellcolor{default2} 0& \cellcolor{default2} 0\\
	\cellcolor{default1} PALMER1D& \cellcolor{default1} False& \cellcolor{default1} 7& \cellcolor{default1} True& \cellcolor{default1} True& \cellcolor{best} 6.526740E-01& \cellcolor{ok} 6.526740E-01& \cellcolor{best} 1& \cellcolor{best} 1& \cellcolor{default1} 0& \cellcolor{default1} 0\\
	\cellcolor{default2} PALMER1E& \cellcolor{default2} True& \cellcolor{default2} 8& \cellcolor{default2} True& \cellcolor{default2} True& \cellcolor{poor} 2.447114E+00& \cellcolor{best} 8.353481E-04& \cellcolor{poor} 227& \cellcolor{best} 43& \cellcolor{default2} 0& \cellcolor{default2} 0\\
	\cellcolor{default1} PALMER2& \cellcolor{default1} True& \cellcolor{default1} 4& \cellcolor{default1} True& \cellcolor{default1} True& \cellcolor{best} 3.651098E+03& \cellcolor{ok} 3.651098E+03& \cellcolor{best} 10& \cellcolor{poor} 902& \cellcolor{default1} 0& \cellcolor{default1} 0\\
	\cellcolor{default2} PALMER2A& \cellcolor{default2} True& \cellcolor{default2} 6& \cellcolor{default2} True& \cellcolor{default2} True& \cellcolor{best} 1.710972E-02& \cellcolor{ok} 1.710972E-02& \cellcolor{best} 76& \cellcolor{ok} 87& \cellcolor{default2} 0& \cellcolor{default2} 0\\
	\cellcolor{default1} PALMER2B& \cellcolor{default1} True& \cellcolor{default1} 4& \cellcolor{default1} True& \cellcolor{default1} True& \cellcolor{ok} 6.232669E-01& \cellcolor{best} 6.232669E-01& \cellcolor{best} 11& \cellcolor{ok} 18& \cellcolor{default1} 0& \cellcolor{default1} 0\\
	\cellcolor{default2} PALMER2C& \cellcolor{default2} False& \cellcolor{default2} 8& \cellcolor{default2} True& \cellcolor{default2} True& \cellcolor{best} 1.436889E-02& \cellcolor{ok} 1.436889E-02& \cellcolor{ok} 2& \cellcolor{best} 1& \cellcolor{default2} 0& \cellcolor{default2} 0\\
	\cellcolor{default1} PALMER2E& \cellcolor{default1} True& \cellcolor{default1} 8& \cellcolor{default1} True& \cellcolor{default1} True& \cellcolor{best} 2.065035E-04& \cellcolor{ok} 2.065044E-04& \cellcolor{poor} 65& \cellcolor{best} 17& \cellcolor{default1} 0& \cellcolor{default1} 0\\
	\cellcolor{default2} PALMER3& \cellcolor{default2} True& \cellcolor{default2} 4& \cellcolor{default2} True& \cellcolor{default2} True& \cellcolor{ok} 2.265958E+03& \cellcolor{best} 2.265958E+03& \cellcolor{best} 11& \cellcolor{poor} 167& \cellcolor{default2} 0& \cellcolor{default2} 0\\
	\cellcolor{default1} PALMER3A& \cellcolor{default1} True& \cellcolor{default1} 6& \cellcolor{default1} True& \cellcolor{default1} True& \cellcolor{best} 2.043143E-02& \cellcolor{ok} 2.043143E-02& \cellcolor{best} 67& \cellcolor{ok} 80& \cellcolor{default1} 0& \cellcolor{default1} 0\\
	\cellcolor{default2} PALMER3B& \cellcolor{default2} True& \cellcolor{default2} 4& \cellcolor{default2} True& \cellcolor{default2} True& \cellcolor{ok} 4.227647E+00& \cellcolor{best} 4.227647E+00& \cellcolor{best} 11& \cellcolor{ok} 14& \cellcolor{default2} 0& \cellcolor{default2} 0\\
	\cellcolor{default1} PALMER3C& \cellcolor{default1} False& \cellcolor{default1} 8& \cellcolor{default1} True& \cellcolor{default1} True& \cellcolor{best} 1.953764E-02& \cellcolor{ok} 1.953764E-02& \cellcolor{best} 1& \cellcolor{best} 1& \cellcolor{default1} 0& \cellcolor{default1} 0\\
	\cellcolor{default2} PALMER3E& \cellcolor{default2} True& \cellcolor{default2} 8& \cellcolor{default2} False& \cellcolor{default2} True& \cellcolor{poor} 3.248139E-02& \cellcolor{best} 5.074119E-05& \cellcolor{poor} 5001& \cellcolor{best} 29& \cellcolor{default2} 1& \cellcolor{default2} 0\\
	\cellcolor{default1} PALMER4& \cellcolor{default1} True& \cellcolor{default1} 4& \cellcolor{default1} True& \cellcolor{default1} True& \cellcolor{ok} 2.285383E+03& \cellcolor{best} 2.285383E+03& \cellcolor{best} 15& \cellcolor{poor} 329& \cellcolor{default1} 0& \cellcolor{default1} 0\\
	\cellcolor{default2} PALMER4A& \cellcolor{default2} True& \cellcolor{default2} 6& \cellcolor{default2} True& \cellcolor{default2} True& \cellcolor{ok} 4.060614E-02& \cellcolor{best} 4.060614E-02& \cellcolor{best} 43& \cellcolor{ok} 56& \cellcolor{default2} 0& \cellcolor{default2} 0\\
	\cellcolor{default1} PALMER4B& \cellcolor{default1} True& \cellcolor{default1} 4& \cellcolor{default1} True& \cellcolor{default1} True& \cellcolor{best} 6.835139E+00& \cellcolor{ok} 6.835139E+00& \cellcolor{best} 12& \cellcolor{ok} 15& \cellcolor{default1} 0& \cellcolor{default1} 0\\
	\cellcolor{default2} PALMER4C& \cellcolor{default2} False& \cellcolor{default2} 8& \cellcolor{default2} True& \cellcolor{default2} True& \cellcolor{best} 5.031069E-02& \cellcolor{ok} 5.031069E-02& \cellcolor{best} 1& \cellcolor{best} 1& \cellcolor{default2} 0& \cellcolor{default2} 0\\
	\cellcolor{default1} PALMER4E& \cellcolor{default1} True& \cellcolor{default1} 8& \cellcolor{default1} True& \cellcolor{default1} True& \cellcolor{poor} 3.708669E-01& \cellcolor{best} 1.480035E-04& \cellcolor{poor} 292& \cellcolor{best} 30& \cellcolor{default1} 0& \cellcolor{default1} 0\\
	\cellcolor{default2} PALMER5A& \cellcolor{default2} True& \cellcolor{default2} 8& \cellcolor{default2} False& \cellcolor{default2} False& \cellcolor{poor} 1.248709E-01& \cellcolor{best} 3.884813E-02& \cellcolor{ok} 5001& \cellcolor{best} 3000& \cellcolor{default2} 1& \cellcolor{default2} 1\\
	\cellcolor{default1} PALMER5B& \cellcolor{default1} True& \cellcolor{default1} 9& \cellcolor{default1} False& \cellcolor{default1} True& \cellcolor{poor} 7.685521E-02& \cellcolor{best} 9.752421E-03& \cellcolor{poor} 5001& \cellcolor{best} 81& \cellcolor{default1} 1& \cellcolor{default1} 0\\
	\cellcolor{default2} PALMER5C& \cellcolor{default2} False& \cellcolor{default2} 6& \cellcolor{default2} True& \cellcolor{default2} True& \cellcolor{best} 2.128087E+00& \cellcolor{ok} 2.128087E+00& \cellcolor{best} 1& \cellcolor{best} 1& \cellcolor{default2} 0& \cellcolor{default2} 0\\
	\cellcolor{default1} PALMER5D& \cellcolor{default1} False& \cellcolor{default1} 4& \cellcolor{default1} True& \cellcolor{default1} True& \cellcolor{ok} 8.733939E+01& \cellcolor{best} 8.733939E+01& \cellcolor{best} 1& \cellcolor{best} 1& \cellcolor{default1} 0& \cellcolor{default1} 0\\
	\cellcolor{default2} PALMER5E& \cellcolor{default2} True& \cellcolor{default2} 8& \cellcolor{default2} True& \cellcolor{default2} False& \cellcolor{poor} 4.524551E-02& \cellcolor{best} 2.088286E-02& \cellcolor{best} 139& \cellcolor{poor} 3000& \cellcolor{default2} 0& \cellcolor{default2} 1\\
	\cellcolor{default1} PALMER6A& \cellcolor{default1} True& \cellcolor{default1} 6& \cellcolor{default1} True& \cellcolor{default1} True& \cellcolor{ok} 5.594885E-02& \cellcolor{best} 5.594885E-02& \cellcolor{best} 99& \cellcolor{ok} 123& \cellcolor{default1} 0& \cellcolor{default1} 0\\
	\cellcolor{default2} PALMER6C& \cellcolor{default2} False& \cellcolor{default2} 8& \cellcolor{default2} False& \cellcolor{default2} True& \cellcolor{poor} 8.478351E-02& \cellcolor{best} 1.638744E-02& \cellcolor{poor} 5001& \cellcolor{best} 1& \cellcolor{default2} 1& \cellcolor{default2} 0\\
	\cellcolor{default1} PALMER6E& \cellcolor{default1} True& \cellcolor{default1} 8& \cellcolor{default1} False& \cellcolor{default1} True& \cellcolor{poor} 3.037913E-02& \cellcolor{best} 2.239541E-04& \cellcolor{poor} 5001& \cellcolor{best} 30& \cellcolor{default1} 1& \cellcolor{default1} 0\\
	\cellcolor{default2} PALMER7A& \cellcolor{default2} True& \cellcolor{default2} 6& \cellcolor{default2} False& \cellcolor{default2} False& \cellcolor{poor} 1.062846E+01& \cellcolor{best} 1.033491E+01& \cellcolor{ok} 5001& \cellcolor{best} 3000& \cellcolor{default2} 1& \cellcolor{default2} 1\\
	\cellcolor{default1} PALMER7C& \cellcolor{default1} False& \cellcolor{default1} 8& \cellcolor{default1} False& \cellcolor{default1} True& \cellcolor{poor} 2.443819E+00& \cellcolor{best} 6.019872E-01& \cellcolor{poor} 5001& \cellcolor{best} 1& \cellcolor{default1} 1& \cellcolor{default1} 0\\
	\cellcolor{default2} PALMER7E& \cellcolor{default2} True& \cellcolor{default2} 8& \cellcolor{default2} False& \cellcolor{default2} False& \cellcolor{poor} 7.851533E+00& \cellcolor{best} 6.574223E+00& \cellcolor{ok} 5001& \cellcolor{best} 3000& \cellcolor{default2} 1& \cellcolor{default2} 1\\
	\cellcolor{default1} PALMER8A& \cellcolor{default1} True& \cellcolor{default1} 6& \cellcolor{default1} True& \cellcolor{default1} True& \cellcolor{best} 7.400970E-02& \cellcolor{ok} 7.400970E-02& \cellcolor{best} 34& \cellcolor{ok} 45& \cellcolor{default1} 0& \cellcolor{default1} 0\\
	\cellcolor{default2} PALMER8C& \cellcolor{default2} False& \cellcolor{default2} 8& \cellcolor{default2} False& \cellcolor{default2} True& \cellcolor{poor} 4.319838E-01& \cellcolor{best} 1.597678E-01& \cellcolor{poor} 5001& \cellcolor{best} 1& \cellcolor{default2} 1& \cellcolor{default2} 0\\
	\cellcolor{default1} PALMER8E& \cellcolor{default1} True& \cellcolor{default1} 8& \cellcolor{default1} True& \cellcolor{default1} True& \cellcolor{ok} 6.339306E-03& \cellcolor{best} 6.339306E-03& \cellcolor{poor} 3582& \cellcolor{best} 23& \cellcolor{default1} 0& \cellcolor{default1} 0\\
	\cellcolor{default2} PARKCH& \cellcolor{default2} False& \cellcolor{default2} 15& \cellcolor{default2} True& \cellcolor{default2} True& \cellcolor{best} -8.974697E+06& \cellcolor{poor} 1.623743E+03& \cellcolor{best} 5& \cellcolor{poor} 17& \cellcolor{default2} 0& \cellcolor{default2} 0\\
	\cellcolor{default1} PENALTY1& \cellcolor{default1} False& \cellcolor{default1} 1000& \cellcolor{default1} False& \cellcolor{default1} True& \cellcolor{poor} 1.114448E+17& \cellcolor{best} 6.439498E+00& \cellcolor{best} 0& \cellcolor{poor} 23& \cellcolor{default1} 2& \cellcolor{default1} 0\\
	\cellcolor{default2} PENALTY2& \cellcolor{default2} False& \cellcolor{default2} 200& \cellcolor{default2} True& \cellcolor{default2} True& \cellcolor{best} 4.711628E+13& \cellcolor{ok} 4.711628E+13& \cellcolor{best} 10& \cellcolor{best} 10& \cellcolor{default2} 0& \cellcolor{default2} 0\\
	\cellcolor{default1} PENTDI& \cellcolor{default1} True& \cellcolor{default1} 5000& \cellcolor{default1} True& \cellcolor{default1} True& \cellcolor{ok} -7.500000E-01& \cellcolor{best} -7.500176E-01& \cellcolor{best} 1& \cellcolor{poor} 15& \cellcolor{default1} 0& \cellcolor{default1} 0\\
	\cellcolor{default2} PFIT1LS& \cellcolor{default2} True& \cellcolor{default2} 3& \cellcolor{default2} True& \cellcolor{default2} True& \cellcolor{ok} 1.244752E-07& \cellcolor{best} 9.577731E-16& \cellcolor{best} 212& \cellcolor{ok} 307& \cellcolor{default2} 0& \cellcolor{default2} 0\\
	\cellcolor{default1} PFIT2LS& \cellcolor{default1} True& \cellcolor{default1} 3& \cellcolor{default1} True& \cellcolor{default1} True& \cellcolor{ok} 8.160119E-09& \cellcolor{best} 1.623231E-16& \cellcolor{best} 64& \cellcolor{ok} 102& \cellcolor{default1} 0& \cellcolor{default1} 0\\
	\cellcolor{default2} PFIT3LS& \cellcolor{default2} True& \cellcolor{default2} 3& \cellcolor{default2} True& \cellcolor{default2} True& \cellcolor{ok} 9.000000E-09& \cellcolor{best} 2.799655E-15& \cellcolor{ok} 146& \cellcolor{best} 141& \cellcolor{default2} 0& \cellcolor{default2} 0\\
	\cellcolor{default1} PFIT4LS& \cellcolor{default1} True& \cellcolor{default1} 3& \cellcolor{default1} True& \cellcolor{default1} True& \cellcolor{ok} 4.613960E-09& \cellcolor{best} 4.404745E-16& \cellcolor{ok} 245& \cellcolor{best} 235& \cellcolor{default1} 0& \cellcolor{default1} 0\\
	\cellcolor{default2} POWELLBC& \cellcolor{default2} True& \cellcolor{default2} 1000& \cellcolor{default2} False& \cellcolor{default2} False& \cellcolor{poor} 3.104193E+05& \cellcolor{best} 0.000000E+00& \cellcolor{best} 1293& \cellcolor{err} None& \cellcolor{default2} 2& \cellcolor{default2} 3\\
	\cellcolor{default1} POWELLBSLS& \cellcolor{default1} False& \cellcolor{default1} 2& \cellcolor{default1} True& \cellcolor{default1} True& \cellcolor{ok} 6.558819E-07& \cellcolor{best} 2.562307E-26& \cellcolor{best} 30& \cellcolor{poor} 91& \cellcolor{default1} 0& \cellcolor{default1} 0\\
	\cellcolor{default2} POWELLSG& \cellcolor{default2} False& \cellcolor{default2} 5000& \cellcolor{default2} True& \cellcolor{default2} True& \cellcolor{ok} 4.204860E-08& \cellcolor{best} 8.332977E-09& \cellcolor{best} 19& \cellcolor{best} 19& \cellcolor{default2} 0& \cellcolor{default2} 0\\
	\cellcolor{default1} PROBPENL& \cellcolor{default1} True& \cellcolor{default1} 500& \cellcolor{default1} True& \cellcolor{default1} True& \cellcolor{ok} 3.991981E-07& \cellcolor{best} 3.981010E-07& \cellcolor{best} 3& \cellcolor{ok} 5& \cellcolor{default1} 0& \cellcolor{default1} 0\\
	\cellcolor{default2} PSPDOC& \cellcolor{default2} True& \cellcolor{default2} 4& \cellcolor{default2} True& \cellcolor{default2} True& \cellcolor{best} 2.414214E+00& \cellcolor{ok} 2.414214E+00& \cellcolor{best} 3& \cellcolor{poor} 7& \cellcolor{default2} 0& \cellcolor{default2} 0\\
	\cellcolor{default1} QR3DLS& \cellcolor{default1} True& \cellcolor{default1} 610& \cellcolor{default1} False& \cellcolor{default1} True& \cellcolor{poor} 7.554681E-02& \cellcolor{best} 5.515677E-16& \cellcolor{ok} 269& \cellcolor{best} 203& \cellcolor{default1} 2& \cellcolor{default1} 0\\
	\cellcolor{default2} QRTQUAD& \cellcolor{default2} True& \cellcolor{default2} 5000& \cellcolor{default2} False& \cellcolor{default2} True& \cellcolor{poor} -4.036489E+10& \cellcolor{best} -2.648567E+11& \cellcolor{best} 13& \cellcolor{poor} 376& \cellcolor{default2} 2& \cellcolor{default2} 0\\
	\cellcolor{default1} QUARTC& \cellcolor{default1} False& \cellcolor{default1} 5000& \cellcolor{default1} False& \cellcolor{default1} True& \cellcolor{poor} 6.240630E+17& \cellcolor{best} 3.935732E+01& \cellcolor{best} 0& \cellcolor{poor} 23& \cellcolor{default1} 2& \cellcolor{default1} 0\\
	\cellcolor{default2} QUDLIN& \cellcolor{default2} True& \cellcolor{default2} 5000& \cellcolor{default2} True& \cellcolor{default2} True& \cellcolor{best} -1.250000E+09& \cellcolor{best} -1.250000E+09& \cellcolor{best} 4& \cellcolor{poor} 28& \cellcolor{default2} 0& \cellcolor{default2} 0\\
	\cellcolor{default1} RAT42LS& \cellcolor{default1} False& \cellcolor{default1} 3& \cellcolor{default1} False& \cellcolor{default1} True& \cellcolor{poor} 1.991585E+04& \cellcolor{best} 8.056523E+00& \cellcolor{best} 0& \cellcolor{poor} 28& \cellcolor{default1} 7& \cellcolor{default1} 0\\
	\cellcolor{default2} RAT43LS& \cellcolor{default2} False& \cellcolor{default2} 4& \cellcolor{default2} True& \cellcolor{default2} True& \cellcolor{poor} 1.076462E+06& \cellcolor{best} 8.786405E+03& \cellcolor{poor} 101& \cellcolor{best} 34& \cellcolor{default2} 0& \cellcolor{default2} 0\\
	\cellcolor{default1} RAYBENDL& \cellcolor{default1} True& \cellcolor{default1} 2050& \cellcolor{default1} False& \cellcolor{default1} False& \cellcolor{poor} 9.786530E+01& \cellcolor{best} -2.872387E+09& \cellcolor{poor} 152& \cellcolor{best} 36& \cellcolor{default1} 2& \cellcolor{default1} 4\\
	\cellcolor{default2} RAYBENDS& \cellcolor{default2} True& \cellcolor{default2} 2050& \cellcolor{default2} False& \cellcolor{default2} False& \cellcolor{poor} 9.789312E+01& \cellcolor{best} -2.913347E+09& \cellcolor{poor} 122& \cellcolor{best} 23& \cellcolor{default2} 2& \cellcolor{default2} 4\\
	\cellcolor{default1} ROSENBR& \cellcolor{default1} False& \cellcolor{default1} 2& \cellcolor{default1} True& \cellcolor{default1} True& \cellcolor{ok} 2.124604E-18& \cellcolor{best} 3.743976E-21& \cellcolor{best} 21& \cellcolor{best} 21& \cellcolor{default1} 0& \cellcolor{default1} 0\\
	\cellcolor{default2} ROSZMAN1LS& \cellcolor{default2} False& \cellcolor{default2} 4& \cellcolor{default2} False& \cellcolor{default2} True& \cellcolor{poor} 1.537246E-01& \cellcolor{best} 4.948485E-04& \cellcolor{poor} 5001& \cellcolor{best} 28& \cellcolor{default2} 1& \cellcolor{default2} 0\\
	\cellcolor{default1} S308& \cellcolor{default1} False& \cellcolor{default1} 2& \cellcolor{default1} True& \cellcolor{default1} True& \cellcolor{best} 7.731991E-01& \cellcolor{ok} 7.731991E-01& \cellcolor{best} 9& \cellcolor{best} 9& \cellcolor{default1} 0& \cellcolor{default1} 0\\
	\cellcolor{default2} S368& \cellcolor{default2} True& \cellcolor{default2} 8& \cellcolor{default2} True& \cellcolor{default2} True& \cellcolor{ok} -7.500000E-01& \cellcolor{best} -7.500000E-01& \cellcolor{poor} 51& \cellcolor{best} 11& \cellcolor{default2} 0& \cellcolor{default2} 0\\
	\cellcolor{default1} SANTALS& \cellcolor{default1} True& \cellcolor{default1} 21& \cellcolor{default1} True& \cellcolor{default1} True& \cellcolor{ok} 1.224358E-05& \cellcolor{best} 1.224358E-05& \cellcolor{ok} 48& \cellcolor{best} 33& \cellcolor{default1} 0& \cellcolor{default1} 0\\
	\cellcolor{default2} SBRYBND& \cellcolor{default2} False& \cellcolor{default2} 5000& \cellcolor{default2} True& \cellcolor{default2} True& \cellcolor{best} 8.311581E-27& \cellcolor{ok} 1.173534E-20& \cellcolor{best} 11& \cellcolor{ok} 13& \cellcolor{default2} 0& \cellcolor{default2} 0\\
	\cellcolor{default1} SCHMVETT& \cellcolor{default1} False& \cellcolor{default1} 5000& \cellcolor{default1} True& \cellcolor{default1} True& \cellcolor{best} -1.499400E+04& \cellcolor{best} -1.499400E+04& \cellcolor{best} 3& \cellcolor{best} 3& \cellcolor{default1} 0& \cellcolor{default1} 0\\
	\cellcolor{default2} SCOSINE& \cellcolor{default2} False& \cellcolor{default2} 5000& \cellcolor{default2} False& \cellcolor{default2} True& \cellcolor{poor} 4.360693E+03& \cellcolor{best} -4.999000E+03& \cellcolor{best} 1& \cellcolor{poor} 129& \cellcolor{default2} 8& \cellcolor{default2} 0\\
	\cellcolor{default1} SENSORS& \cellcolor{default1} False& \cellcolor{default1} 100& \cellcolor{default1} True& \cellcolor{default1} True& \cellcolor{poor} -1.919531E+03& \cellcolor{best} -1.987875E+03& \cellcolor{best} 24& \cellcolor{ok} 36& \cellcolor{default1} 0& \cellcolor{default1} 0\\
	\cellcolor{default2} SIM2BQP& \cellcolor{default2} True& \cellcolor{default2} 2& \cellcolor{default2} True& \cellcolor{default2} True& \cellcolor{ok} 0.000000E+00& \cellcolor{best} -7.471066E-09& \cellcolor{best} 1& \cellcolor{poor} 7& \cellcolor{default2} 0& \cellcolor{default2} 0\\
	\cellcolor{default1} SIMBQP& \cellcolor{default1} True& \cellcolor{default1} 2& \cellcolor{default1} True& \cellcolor{default1} True& \cellcolor{ok} 0.000000E+00& \cellcolor{best} -7.421861E-09& \cellcolor{best} 1& \cellcolor{poor} 7& \cellcolor{default1} 0& \cellcolor{default1} 0\\
	\cellcolor{default2} SINEALI& \cellcolor{default2} True& \cellcolor{default2} 1000& \cellcolor{default2} True& \cellcolor{default2} True& \cellcolor{ok} -9.989947E+04& \cellcolor{best} -9.990096E+04& \cellcolor{best} 9& \cellcolor{poor} 26& \cellcolor{default2} 0& \cellcolor{default2} 0\\
	\cellcolor{default1} SINEVAL& \cellcolor{default1} False& \cellcolor{default1} 2& \cellcolor{default1} True& \cellcolor{default1} True& \cellcolor{ok} 1.450661E-21& \cellcolor{best} 5.787363E-43& \cellcolor{best} 42& \cellcolor{best} 42& \cellcolor{default1} 0& \cellcolor{default1} 0\\
	\cellcolor{default2} SINQUAD& \cellcolor{default2} False& \cellcolor{default2} 5000& \cellcolor{default2} False& \cellcolor{default2} True& \cellcolor{ok} -6.757014E+06& \cellcolor{best} -6.757014E+06& \cellcolor{best} 10& \cellcolor{poor} 34& \cellcolor{default2} 5& \cellcolor{default2} 0\\
	\cellcolor{default1} SISSER& \cellcolor{default1} False& \cellcolor{default1} 2& \cellcolor{default1} True& \cellcolor{default1} True& \cellcolor{ok} 2.105255E-09& \cellcolor{best} 6.331104E-13& \cellcolor{best} 13& \cellcolor{ok} 18& \cellcolor{default1} 0& \cellcolor{default1} 0\\
	\cellcolor{default2} SNAIL& \cellcolor{default2} False& \cellcolor{default2} 2& \cellcolor{default2} True& \cellcolor{default2} True& \cellcolor{best} 2.431911E-29& \cellcolor{ok} 1.472743E-28& \cellcolor{ok} 67& \cellcolor{best} 63& \cellcolor{default2} 0& \cellcolor{default2} 0\\
	\cellcolor{default1} SPARSINE& \cellcolor{default1} False& \cellcolor{default1} 5000& \cellcolor{default1} False& \cellcolor{default1} True& \cellcolor{poor} 2.147088E+06& \cellcolor{best} 1.295293E-07& \cellcolor{ok} 19& \cellcolor{best} 15& \cellcolor{default1} 2& \cellcolor{default1} 0\\
	\cellcolor{default2} SPECAN& \cellcolor{default2} True& \cellcolor{default2} 9& \cellcolor{default2} True& \cellcolor{default2} True& \cellcolor{best} 1.645655E-13& \cellcolor{ok} 2.306852E-13& \cellcolor{best} 9& \cellcolor{ok} 10& \cellcolor{default2} 0& \cellcolor{default2} 0\\
	\cellcolor{default1} SPMSRTLS& \cellcolor{default1} False& \cellcolor{default1} 4999& \cellcolor{default1} False& \cellcolor{default1} True& \cellcolor{ok} 3.192705E-08& \cellcolor{best} 1.855581E-15& \cellcolor{best} 14& \cellcolor{ok} 22& \cellcolor{default1} 2& \cellcolor{default1} 0\\
	\cellcolor{default2} SROSENBR& \cellcolor{default2} False& \cellcolor{default2} 5000& \cellcolor{default2} True& \cellcolor{default2} True& \cellcolor{best} 2.681252E-23& \cellcolor{ok} 3.301788E-22& \cellcolor{ok} 9& \cellcolor{best} 8& \cellcolor{default2} 0& \cellcolor{default2} 0\\
	\cellcolor{default1} SSBRYBND& \cellcolor{default1} False& \cellcolor{default1} 5000& \cellcolor{default1} True& \cellcolor{default1} True& \cellcolor{best} 6.519279E-26& \cellcolor{ok} 7.710062E-13& \cellcolor{best} 10& \cellcolor{poor} 26& \cellcolor{default1} 0& \cellcolor{default1} 0\\
	\cellcolor{default2} SSCOSINE& \cellcolor{default2} False& \cellcolor{default2} 5000& \cellcolor{default2} False& \cellcolor{default2} True& \cellcolor{poor} 4.306404E+03& \cellcolor{best} -4.999000E+03& \cellcolor{best} 3& \cellcolor{poor} 71& \cellcolor{default2} 5& \cellcolor{default2} 0\\
	\cellcolor{default1} SSI& \cellcolor{default1} False& \cellcolor{default1} 3& \cellcolor{default1} True& \cellcolor{default1} False& \cellcolor{ok} 1.652794E-05& \cellcolor{best} 1.385069E-09& \cellcolor{best} 142& \cellcolor{poor} 3000& \cellcolor{default1} 0& \cellcolor{default1} 1\\
	\cellcolor{default2} STRATEC& \cellcolor{default2} False& \cellcolor{default2} 10& \cellcolor{default2} True& \cellcolor{default2} True& \cellcolor{best} -2.726163E+07& \cellcolor{poor} 2.212262E+03& \cellcolor{best} 2& \cellcolor{poor} 24& \cellcolor{default2} 0& \cellcolor{default2} 0\\
	\cellcolor{default1} TESTQUAD& \cellcolor{default1} False& \cellcolor{default1} 5000& \cellcolor{default1} True& \cellcolor{default1} True& \cellcolor{best} 0.000000E+00& \cellcolor{best} 0.000000E+00& \cellcolor{best} 1& \cellcolor{best} 1& \cellcolor{default1} 0& \cellcolor{default1} 0\\
	\cellcolor{default2} THURBERLS& \cellcolor{default2} False& \cellcolor{default2} 7& \cellcolor{default2} True& \cellcolor{default2} True& \cellcolor{ok} 5.642708E+03& \cellcolor{best} 5.642708E+03& \cellcolor{ok} 21& \cellcolor{best} 19& \cellcolor{default2} 0& \cellcolor{default2} 0\\
	\cellcolor{default1} TOINTGOR& \cellcolor{default1} False& \cellcolor{default1} 50& \cellcolor{default1} True& \cellcolor{default1} True& \cellcolor{ok} 1.373905E+03& \cellcolor{best} 1.373905E+03& \cellcolor{best} 7& \cellcolor{best} 7& \cellcolor{default1} 0& \cellcolor{default1} 0\\
	\cellcolor{default2} TOINTGSS& \cellcolor{default2} False& \cellcolor{default2} 5000& \cellcolor{default2} True& \cellcolor{default2} True& \cellcolor{best} 1.000000E+01& \cellcolor{ok} 1.000000E+01& \cellcolor{best} 1& \cellcolor{best} 1& \cellcolor{default2} 0& \cellcolor{default2} 0\\
	\cellcolor{default1} TOINTPSP& \cellcolor{default1} False& \cellcolor{default1} 50& \cellcolor{default1} True& \cellcolor{default1} True& \cellcolor{ok} 2.255604E+02& \cellcolor{best} 2.255604E+02& \cellcolor{best} 13& \cellcolor{ok} 20& \cellcolor{default1} 0& \cellcolor{default1} 0\\
	\cellcolor{default2} TOINTQOR& \cellcolor{default2} False& \cellcolor{default2} 50& \cellcolor{default2} True& \cellcolor{default2} True& \cellcolor{ok} 1.175472E+03& \cellcolor{best} 1.175472E+03& \cellcolor{best} 1& \cellcolor{best} 1& \cellcolor{default2} 0& \cellcolor{default2} 0\\
	\cellcolor{default1} TQUARTIC& \cellcolor{default1} False& \cellcolor{default1} 5000& \cellcolor{default1} True& \cellcolor{default1} True& \cellcolor{best} 5.458963E-23& \cellcolor{ok} 1.804881E-22& \cellcolor{best} 1& \cellcolor{best} 1& \cellcolor{default1} 0& \cellcolor{default1} 0\\
	\cellcolor{default2} TRIDIA& \cellcolor{default2} False& \cellcolor{default2} 5000& \cellcolor{default2} True& \cellcolor{default2} True& \cellcolor{ok} 6.491900E-25& \cellcolor{best} 6.345166E-25& \cellcolor{best} 1& \cellcolor{best} 1& \cellcolor{default2} 0& \cellcolor{default2} 0\\
	\cellcolor{default1} VARDIM& \cellcolor{default1} False& \cellcolor{default1} 200& \cellcolor{default1} False& \cellcolor{default1} True& \cellcolor{poor} 3.256542E+16& \cellcolor{best} 1.743420E-06& \cellcolor{best} 0& \cellcolor{poor} 27& \cellcolor{default1} 2& \cellcolor{default1} 0\\
	\cellcolor{default2} VAREIGVL& \cellcolor{default2} False& \cellcolor{default2} 50& \cellcolor{default2} True& \cellcolor{default2} True& \cellcolor{ok} 4.234969E-10& \cellcolor{best} 1.155963E-19& \cellcolor{best} 11& \cellcolor{ok} 13& \cellcolor{default2} 0& \cellcolor{default2} 0\\
	\cellcolor{default1} VESUVIALS& \cellcolor{default1} False& \cellcolor{default1} 8& \cellcolor{default1} False& \cellcolor{default1} True& \cellcolor{poor} 1.978977E+03& \cellcolor{best} 9.914100E+02& \cellcolor{poor} 5001& \cellcolor{best} 48& \cellcolor{default1} 1& \cellcolor{default1} 0\\
	\cellcolor{default2} VESUVIOLS& \cellcolor{default2} False& \cellcolor{default2} 8& \cellcolor{default2} True& \cellcolor{default2} True& \cellcolor{ok} 9.914100E+02& \cellcolor{best} 9.914100E+02& \cellcolor{best} 8& \cellcolor{ok} 10& \cellcolor{default2} 0& \cellcolor{default2} 0\\
	\cellcolor{default1} VESUVIOULS& \cellcolor{default1} False& \cellcolor{default1} 8& \cellcolor{default1} True& \cellcolor{default1} True& \cellcolor{best} 4.771138E-01& \cellcolor{ok} 4.771138E-01& \cellcolor{best} 8& \cellcolor{best} 8& \cellcolor{default1} 0& \cellcolor{default1} 0\\
	\cellcolor{default2} VIBRBEAM& \cellcolor{default2} False& \cellcolor{default2} 8& \cellcolor{default2} True& \cellcolor{default2} True& \cellcolor{poor} 1.010408E+01& \cellcolor{best} 3.322376E-01& \cellcolor{best} 19& \cellcolor{poor} 58& \cellcolor{default2} 0& \cellcolor{default2} 0\\
	\cellcolor{default1} WALL10& \cellcolor{default1} True& \cellcolor{default1} 1461& \cellcolor{default1} False& \cellcolor{default1} True& \cellcolor{poor} 1.533520E+01& \cellcolor{best} -4.559538E+05& \cellcolor{poor} 528& \cellcolor{best} 32& \cellcolor{default1} 2& \cellcolor{default1} 0\\
	\cellcolor{default2} WATSON& \cellcolor{default2} False& \cellcolor{default2} 12& \cellcolor{default2} True& \cellcolor{default2} True& \cellcolor{ok} 8.658558E-06& \cellcolor{best} 2.131709E-09& \cellcolor{poor} 110& \cellcolor{best} 21& \cellcolor{default2} 0& \cellcolor{default2} 0\\
	\cellcolor{default1} WEEDS& \cellcolor{default1} True& \cellcolor{default1} 3& \cellcolor{default1} True& \cellcolor{default1} True& \cellcolor{ok} 2.587277E+00& \cellcolor{best} 2.587277E+00& \cellcolor{best} 19& \cellcolor{ok} 25& \cellcolor{default1} 0& \cellcolor{default1} 0\\
	\cellcolor{default2} WOODS& \cellcolor{default2} False& \cellcolor{default2} 4000& \cellcolor{default2} True& \cellcolor{default2} True& \cellcolor{best} 1.233828E-27& \cellcolor{ok} 4.837167E-24& \cellcolor{ok} 42& \cellcolor{best} 40& \cellcolor{default2} 0& \cellcolor{default2} 0\\
	\cellcolor{default1} YATP1LS& \cellcolor{default1} False& \cellcolor{default1} 2600& \cellcolor{default1} True& \cellcolor{default1} True& \cellcolor{ok} 7.353642E-10& \cellcolor{best} 7.795463E-20& \cellcolor{ok} 28& \cellcolor{best} 20& \cellcolor{default1} 0& \cellcolor{default1} 0\\
	\cellcolor{default2} YATP2LS& \cellcolor{default2} False& \cellcolor{default2} 2600& \cellcolor{default2} True& \cellcolor{default2} True& \cellcolor{ok} 7.471263E-10& \cellcolor{best} 2.731200E-28& \cellcolor{best} 19& \cellcolor{ok} 31& \cellcolor{default2} 0& \cellcolor{default2} 0\\
	\cellcolor{default1} YFIT& \cellcolor{default1} True& \cellcolor{default1} 3& \cellcolor{default1} True& \cellcolor{default1} True& \cellcolor{best} 6.669755E-13& \cellcolor{ok} 6.717568E-13& \cellcolor{best} 36& \cellcolor{ok} 49& \cellcolor{default1} 0& \cellcolor{default1} 0\\
	\cellcolor{default2} YFITU& \cellcolor{default2} False& \cellcolor{default2} 3& \cellcolor{default2} True& \cellcolor{default2} True& \cellcolor{ok} 6.669755E-13& \cellcolor{best} 6.669727E-13& \cellcolor{ok} 36& \cellcolor{best} 35& \cellcolor{default2} 0& \cellcolor{default2} 0\\
	\cellcolor{default1} ZANGWIL2& \cellcolor{default1} False& \cellcolor{default1} 2& \cellcolor{default1} True& \cellcolor{default1} True& \cellcolor{best} -1.820000E+01& \cellcolor{ok} -1.820000E+01& \cellcolor{best} 1& \cellcolor{best} 1& \cellcolor{default1} 0& \cellcolor{default1} 0\\
\caption{Results of testing Noontime and Ipopt on the the listed problems from the Cutest problem set.
\textit{name} refers to the name of the problem, \textit{bounds} is \textit{True} if the problem has bounds on the variables, otherwise its \textit{False}, \textit{n} refers to the number of variables of the respective problem, \textit{success} is \textit{True} if the problem was solved, otherwise its \textit{False}. $f$ denotes the objective function value of the found minimum or the objective function value of the last iterate, when the minimization process was aborted. \textit{\#iter} gives the number of iterations until the process terminated and \textit{code} refers to the code of the result message that we encoded in Table~\ref{tab:stat:noontime} and in Table~\ref{tab:stat:ipopt}}
%\label{tab:full_cutest_results}
\end{longtable}
	\normalsize
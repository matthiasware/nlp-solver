\chapter{Measurements and Results}\label{ch:results}
\section{Measurements}
We tested our implementation on a subset of the CUTEst problem set. For this we selected only bounded and unbounded problems with variable-size $2 \leq n \leq 5000$. In total $309$ problems were used for testing. Throughout the test runs, the following solver configuration was employed
\begin{flalign*}
	f_{tol} = 1e^{-8} \; \text{(see~\eqref{eq:ftol}) }
	\qquad
	g_{tol} = 1e^{-8} \; \text{(see~\eqref{eq:gtol})}
\end{flalign*}
We set the limit for the number of iterations at
\begin{flalign*}
	k_{max} = 5000
\end{flalign*}
and confined the CPU runtime of the solving process to $360$ seconds.
The variables of interest were:
\begin{itemize}
	\item \textbf{Success}: Was the problem solved or unsolved.
	\item \textbf{Function value}: The objective function value of the last iterate, which is the minimum if the solver terminated successfully, or the last iterate when the solver was aborted due to the time cap, or the limit on the number of iterations.
	\item \textbf{Iterations}: Number of iterations performed until the solver terminated.
	\item \textbf{Message} The result message of the solver.
\end{itemize}
The summary of the NOONTIME test run is listed in Table~\ref{tab:stat:noontime}. \\

For comparison, we used the open source solver library IPopt (version 3.12.5)  ~\cite{webpage:ipopt}. IPopt as part of the COIN-OR initiative is written in C++ and primarily optimized for large-scale optimization problems.
It was initially released in 2005 and since then steadily improved. It is well recognized in both academics and industry ~\cite{webpage:ipopt}.
This and the fact that IPopt uses Newton's method, makes it a good competitor for NOONTIME.
We ran IPopt with its default configuration on the same problem set.
A summary of the Ipopt test run is listed in Table~\ref{tab:stat:ipopt}. \\

The results of all the test runs for both, NOONTIME and IPopt can be found in Table~\ref{tab:fullresults}. \\

For comparing the results from both of the solvers we use the \textit{relative solver error}. Here we introduce the \textit{relative solver error} on the values of the objective functions, but it is similarly defined and used for the number of iterations. Given a problem from our test set with results for the objective function value from NOONTIME, $f_{nt}$, and from IPopt, $f_{opt}$.
Since the absolute error of both results $|f_{nt} - f_{pt} |$ depends heavily on the problem and does not represent the goodness of the final results very well, we define the \textit{relative solver error} for $f_{nt}$ and $f_{opt}$ as:

\begin{flalign}
err_{rel}(i) = \frac{f_i - f_{min}}{|f_{min}| + 1} \qquad f_{min} = min(f_{opt}, f_{nt}) \quad i \in \{nt, opt\}
\end{flalign}
This allows us to classify the result. With $\epsilon$ being appropriately selected, we denote three classes as:
\begin{flalign}
&\!\begin{aligned}
f_{opt} \ll f_{nt} \quad \Leftrightarrow \quad
& f_{opt} < f_{nt} \text{ and } err_{rel}(nt) > \epsilon \\
f_{opt} \approx f_{nt} \quad \Leftrightarrow \quad
& f_{nt} \leq f_{opt} \text{ and }  err_{rel}(opt) < \epsilon \text{ or } \\
& f_{opt} \leq f_{nt} \text{ and } err_{rel}(nt) < \epsilon \\
f_{opt} \gg f_{nt} \quad \Leftrightarrow \quad 
& f_{opt} < f_{nt} \text{ and } err_{rel}(opt) > \epsilon
\end{aligned}
\label{eq:def:goodness}
\end{flalign}
Informally, these classes can be interpreted verbally as:
\begin{itemize}
	\item $f_{opt} \approx f_{nt}$: IPopt and NOONTIME did equally well.
	\item $f_{opt} \ll f_{nt}$: IPopt did better than NOONTIME.
	\item $f_{opt} \gg f_{nt}$: NOONTIME did better than IPopt.
\end{itemize}

The same holds for the number of iterations of IPopt, $Iter_{opt}$, and NOONTIME, $Iter_{nt}$. For the evaluation of our results, the relative solver error was computed with the following parameters:
\begin{itemize}
	\item Objective function value: $\epsilon = 1e^{-4}$
	\item Number of iterations: $\epsilon = 1$
\end{itemize}
In the full result Table~\ref{tab:fullresults}, we color-encoded these classes for both the number of iterations and the objective function values.
Let $f_{opt} \geq f_{nt}$. Then the cell for $f_{opt}$ is colored \textit{dark green}. If $f_{opt} = f_{nt}$, then also the cell for $f_{nt}$ is colored \textit{dark green}. Otherwise if $f_{nt} \approx f_{opt}$, then it is colored \textit{light green} and if $f_{opt} \ll f_{nt}$, it is colored \textit{orange}.
Informally, a cell is \textit{dark green} if it holds the best result for this specific problem, it is \textit{light green} if the result is similarly good as the best result and it is \textit{orange} if it is sufficiently worse than the best result.
The same rules apply to the columns of \textit{\#iter}.

\begin{table}[H]
	\begin{tabular}{cccl}
		\multicolumn{4}{c}{\textit{NOONTIME}} \\
		\toprule
		\toprule
		\textbf{status} & \textbf{code} & \textbf{count} & \textbf{reason} \\
		\midrule
		solved & 0 & 241 & Optimal Solution Found. \\
		\midrule
		\multirow{6}{*}{unsolved}
		& 1 & 16 & Maximum number of iterations exceeded. \\
		& 2 & 40 & Timeout after 360 seconds. \\
		& 5 & 9 & Invalid iterate encountered ($f^{(k+1)} > f^{(k)}$). \\
		& 6 & 1 & Overflow encountered in double scalars. \\
		& 7 & 1 & Eigenvalues did not converge. \\
		& 8 & 1 & Invalid search direction encountered ($g^T \Delta x > 0)$. \\
		\bottomrule
		& $\Sigma$ & $309$ &  \\
		\bottomrule
	\end{tabular}
	\caption{Result breakdown of running NOONTIME on the CUTEst test set.}
	\label{tab:stat:noontime}
\end{table}

\begin{table}[H]
\begin{tabular}{cccl}
	\multicolumn{4}{c}{\textit{IPopt}} \\
	\toprule
	\toprule
	\textbf{status} & \textbf{code} & \textbf{count} & \textbf{reason} \\
	\midrule
	solved & 0 & 290 & Optimal Solution Found. \\
	\midrule
%	\multirow{4}{*}{\rotatebox[origin=c]{90}{unsolved}}
	\multirow{4}{*}{unsolved}
	& 1 & 11 & Maximum Number of Iterations exceeded. \\
	& 2 & 5 & Timeout after 360 seconds. \\
	& 3 & 1 & Invalid number in NLP function or derivative detected. \\
	& 4 & 2 & Error in step computation. \\
	\bottomrule
	& $\Sigma$  &  $309$ & \\
	\bottomrule
\end{tabular}
\caption{Result breakdown of running IPopt on the CUTEst test set.}
\label{tab:stat:ipopt}
\end{table}

\begin{table}[H]
	\begin{tabular}{ccccc}
		\multicolumn{4}{c}{\textit{Problems solved by NOONTIME and IPopt}} \\
		\toprule
		\toprule
		& \multicolumn{3}{c}{\textbf{Function value}} \\
		\cmidrule{2-4}
		\textbf{Iterations} & $f_{opt} \ll f_{nt}$ & $f_{opt} \approx f_{nt}$  &  $f_{opt} \gg f_{nt}$ & $\Sigma$\\
		\midrule
		%	\multirow{4}{*}{\rotatebox[origin=c]{90}{Iterations}} 
		$Iter_{opt} \ll Iter_{nt}$  & 4 & 27 & 0 & 31 \\
		$Iter_{opt} \approx Iter_{nt}$ & 6 & 160 & 3 & 169 \\
		$Iter_{opt} \gg Iter_{nt}$ & 7 & 24 & 4 & 35 \\
		\midrule
		$\Sigma$ & 17 & 211 & 7 & 235  \\
		\bottomrule
	\end{tabular}
	\caption{Comparing the results from all problems, that both, IPopt and NOONTIME solved.} 
	\label{tab:ipopt_vs_noontime_solved}
\end{table}


\section{Evaluation}
On the given $309$ problems, NOONTIME solved $241$, which amounts a success rate of $77.99 \%$ and IPopt in comparison solved $290$ problems, equivalent to $93.94 \%$. The relationships of results regarding the success variable are listed in Table~\ref{tab:noontime_vs_ipopt_total}.

\begin{table}[H]
	%\noindent
	\begin{center}
		\renewcommand\arraystretch{1.5}
		\setlength\tabcolsep{0pt}
		\begin{tabular}{c >{\bfseries}r @{\hspace{0.7em}}c @{\hspace{0.4em}}c @{\hspace{0.7em}}l}
			\multirow{10}{*}{\parbox{1.1cm}{\bfseries\raggedleft IPopt}} & 
			& \multicolumn{2}{c}{\bfseries NOONTIME} & \\
			& & \bfseries solved & \bfseries unsolved & \bfseries $\Sigma$ \\
			& solved & \MyBoxx{235}{} & \MyBoxx{55}{} & 290 \\[2.4em]
			& unsolved & \MyBoxx{6}{} & \MyBoxx{13}{} & 19 \\
			& $\Sigma$ & 241 & 68 & 309
		\end{tabular}
	\end{center}
	\caption{Results of testing NOONTIME and IPopt on the problem set.}
	\label{tab:noontime_vs_ipopt_total}
\end{table}

\subsection{Unsolved problems}
For NOONTIME, the largest group of unsolved problems is the group with code number $2$ where the solver process was aborted due to the cut-off time. It counts $40$ problems in total. By comparison, for IPopt the same group contains only $5$ members. This gap in performance can be explained easily:
\footnote{We note here, that we did not optimize for speed in terms of CPU time, which is also the reason why we did not measure and compare the runtimes in the results.}
\begin{enumerate}
	\item \textbf{Sparsity}: In contrast to IPopt, NOONTIME does not exploit sparsity structures in the Hessian matrix. Especially for problems with many variables, this slows down the solving process with expensive operations like eigenvalue computation or the solving of linear systems.

	\item \textbf{Native Python}: The Cauchy point computation and the subspace minimization consist of many loops that are implemented in native Python which runs slower in comparison to a compiled version of the same.
\end{enumerate}

The second largest group (code number $1$) of unsolved problems with $16$ members are the problems where the maximum number of iterations was exceeded. Here we are not much worse off than IPopt with $11$ problems in this category. Exploratory test runs with modified solver configuration settings suggests, that the size of this group can be reduced by adjusting the solver configurations to the specific problems. \\

The three problems with exit code 6,7 and 8 in Table~\ref{tab:stat:noontime} could not be solved because of numerical issues in our implementation.
The problems with exit code 5 in Table~\ref{tab:stat:noontime} could not be solved due to numerical issues in the \textit{scipy} line search that we rely on. \\

\subsection{Solved problems}

As shown in Table~\ref{tab:noontime_vs_ipopt_total} there are $235$ problems that both IPopt and NOONTIME solved. We classified the results of these problems according to the criterion~\eqref{eq:def:goodness} which is summarized in Table~\ref{tab:ipopt_vs_noontime_solved}. \\
From the $235$ there are $191$ problems or $81,3\%$ where NOONTIME was at least as good as IPopt, both in terms of the goodness of the minimum and the number of iterations. In contrast, for IPopt there are $197$ problems or $83,9\%$, where it was at least as good as NOONTIME.

Now considering the $211$ problems where $f_{opt} \approx f_{nt}$, we were interested in how fast (in terms of iterations) the two implementations converged. By correlating the relative number of problems solved with the relative overhead in terms of iterations as shown in in Figure~\ref{fig:iterations}, we conclude, that both solvers behave similarly regarding convergence speed.

\begin{figure}[H]
	\includegraphics[width=\textwidth]{img/iterations.png}
	\caption{Convergence speed for the intersection of problems that were solved by IPopt and NOONTIME. The blue curve represents NOONTIME and the orange one IPopt.}
	\label{fig:iterations}
\end{figure}